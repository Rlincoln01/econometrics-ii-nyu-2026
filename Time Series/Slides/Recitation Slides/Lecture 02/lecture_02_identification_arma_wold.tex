\documentclass[aspectratio=169,11pt]{beamer}

% Theme and color setup
\usetheme{default}
\usecolortheme{default}

% Remove navigation symbols
\setbeamertemplate{navigation symbols}{}

% Custom colors (MIT-inspired)
\definecolor{mitred}{RGB}{163,31,52}
\definecolor{darkblue}{RGB}{0,40,85}
\definecolor{lightgray}{RGB}{245,245,245}
\definecolor{darkgray}{RGB}{64,64,64}

% Set main colors
\setbeamercolor{frametitle}{fg=white,bg=darkblue}
\setbeamercolor{title}{fg=darkblue}
\setbeamercolor{author}{fg=black}
\setbeamercolor{institute}{fg=black}
\setbeamercolor{date}{fg=black}
\setbeamercolor{structure}{fg=darkblue}
\setbeamercolor{normal text}{fg=black}
\setbeamercolor{itemize item}{fg=mitred}
\setbeamercolor{itemize subitem}{fg=darkblue}
\setbeamercolor{enumerate item}{fg=darkblue}

% Frame title band
\newlength{\titlebandht}
\setlength{\titlebandht}{4ex}

\setbeamertemplate{frametitle}{
  \nointerlineskip
  \begin{beamercolorbox}[wd=\paperwidth,ht=\titlebandht,dp=1.1ex,leftskip=1.2em,rightskip=1.2em]{frametitle}
    \usebeamerfont{frametitle}\insertframetitle
    \ifx\insertframesubtitle\@empty\relax\else\\[-0.2ex]
      \usebeamerfont{framesubtitle}\insertframesubtitle
    \fi
  \end{beamercolorbox}
  \vspace{0.6em}
}

% Font settings
\usefonttheme{professionalfonts}
\usefonttheme{serif}

% Footline with page numbers
\setbeamertemplate{footline}{
    \hfill\insertframenumber\hspace{2em}\vspace{0.5em}
}

% Packages
\usepackage{amsmath,amssymb,amsthm}
\usepackage{graphicx}
\usepackage{tikz}
\usepackage{booktabs}
\usepackage{mathtools}
\usepackage{bm}
\usepackage{bbm}
\usepackage{hyperref}

% Custom commands for math
\newcommand{\E}{\mathbb{E}}
\renewcommand{\P}{\mathbb{P}}
\newcommand{\R}{\mathbb{R}}
\newcommand{\N}{\mathbb{N}}
\newcommand{\Z}{\mathbb{Z}}
\newcommand{\F}{\mathcal{F}}
\newcommand{\Cov}{\text{Cov}}
\newcommand{\Var}{\text{Var}}
\newcommand{\Corr}{\text{Corr}}
\newcommand{\Span}{\text{span}}
\newcommand{\xbf}{\mathbf{x}}
\newcommand{\ybf}{\mathbf{y}}
\newcommand{\zbf}{\mathbf{z}}
\newcommand{\ubf}{\mathbf{u}}
\newcommand{\vbf}{\mathbf{v}}
\DeclareMathOperator*{\argmin}{arg\,min}
\DeclareMathOperator*{\argmax}{arg\,max}
\DeclareMathOperator*{\plim}{plim}
\DeclareMathOperator{\proj}{proj}

% Theorem environments
\setbeamertemplate{theorems}[numbered]
\theoremstyle{plain}
\newtheorem{proposition}{Proposition}
\newtheorem{assumption}{Assumption}
\theoremstyle{definition}

% Block colors
\setbeamercolor{block title}{bg=lightgray,fg=darkblue}
\setbeamercolor{block body}{bg=lightgray!50}

% Alert block colors
\setbeamercolor{block title alerted}{bg=mitred!15,fg=mitred}
\setbeamercolor{block body alerted}{bg=mitred!5}

% Title information
\title{Identification, ARMA Models, and Wold Decomposition}
\subtitle{TA Session 2: Reduced-Form Time Series Analysis}
\author{Rafael Lincoln}
\institute{Econometrics II -- NYU PhD Program}
\date{Spring 2026}

\begin{document}

% ============================================================================
% TITLE SLIDE
% ============================================================================
\begin{frame}[plain]
    \titlepage
\end{frame}

% ============================================================================
% OUTLINE
% ============================================================================
\begin{frame}{Today's Roadmap}
    \textbf{Hour 1: Theory} (~25-30 minutes per topic)
    \begin{enumerate}
        \item \textbf{Identification in Time Series}
        \begin{itemize}
            \item Global vs. local identification
            \item Observational equivalence and ACGF
            \item Causality \& invertibility
        \end{itemize}
        \vspace{0.3em}
        \item \textbf{Reduced-Form Models: AR, MA, ARMA}
        \begin{itemize}
            \item Definitions, stationarity, and invertibility
            \item Observational equivalence in MA processes
        \end{itemize}
        \vspace{0.3em}
        \item \textbf{Wold's Decomposition Theorem}
    \end{enumerate}
    
    \vspace{0.5em}
    \textbf{Hour 2: Exercises}
    \begin{itemize}
        \item Tim's Practice Problems (Ex. 2, 3)
        \item Medeiros ARMA Exercises (1, 2, 3, 6) + Vector Processes (Ex. 4)
    \end{itemize}
\end{frame}

% ============================================================================
% PART I: IDENTIFICATION
% ============================================================================
\begin{frame}[plain]
    \begin{center}
        {\Large \textbf{Part I: Identification in Time Series}}\\[1em]
        {\large The Challenge of Learning from Data}
    \end{center}
    
    \vspace{2em}
    
    \begin{center}
        \textcolor{mitred}{\textbf{Always Remember: Identification is a non-testable assumption!}}
    \end{center}
\end{frame}

% ----------------------------------------------------------------------------
\begin{frame}{The Identification Problem: Setup}
    \textbf{Setup}: We have a parametric model $\mathcal{M}(\theta)$ with structural parameters $\theta \in \Theta$.
    
    \vspace{0.5em}
    
    The model generates a family of distributions for observed data:
    \[
    \mathbb{D}(\mathbf{Y}^T \mid \theta), \quad \text{where } \mathbf{Y}^T = \{y_t\}_{t=1}^T
    \]
    
    \vspace{0.5em}
    
    \textbf{The Fundamental Question}:
    \begin{center}
        \textit{Can we uniquely determine $\theta$ from the distribution of the data?}
    \end{center}
    
    \vspace{0.5em}
    
    \begin{alertblock}{Why This Matters}
        If two different values of $\theta$ generate the \textbf{same distribution} of data, then we \textbf{cannot distinguish} between them using any amount of data!
    \end{alertblock}
\end{frame}

% ----------------------------------------------------------------------------
\begin{frame}{Global and Local Identification}
    \label{slide:global_local_id}
    \begin{definition}[Global Identification]
        $\theta_0 \in \Theta$ is \textbf{globally identified} if for all $\theta \in \Theta$:
        \[
        \mathbb{D}(\mathbf{Y}^T \mid \theta) = \mathbb{D}(\mathbf{Y}^T \mid \theta_0) \implies \theta = \theta_0
        \]
    \end{definition}
    
    \vspace{0.5em}
    
    \begin{definition}[Local Identification]
        $\theta_0 \in \Theta$ is \textbf{locally identified} if there exists a neighborhood $U_{\theta_0}$ such that for all $\theta \in U_{\theta_0}$:
        \[
        \mathbb{D}(\mathbf{Y}^T \mid \theta) = \mathbb{D}(\mathbf{Y}^T \mid \theta_0) \implies \theta = \theta_0
        \]
    \end{definition}
    
    \vspace{0.3em}
    
    \textbf{Intuition}: Global $\Rightarrow$ unique in entire space; Local $\Rightarrow$ unique in a neighborhood.
    
    \vspace{0.2em}
    
    \hfill \hyperlink{slide:info_matrix}{\beamerbutton{How to check? $\rightarrow$ Appendix}}
\end{frame}

% ----------------------------------------------------------------------------
\begin{frame}{Linear Gaussian Models: Why They Matter for Time Series}
    For \textbf{linear Gaussian models}: $\mathbf{Y}^T \sim \mathcal{N}(\mu(\theta), \Sigma(\theta))$
    
    \vspace{0.5em}
    
    The mean and variance \textbf{fully characterize} the distribution.
    
    \vspace{0.5em}
    
    \begin{block}{Connection to Time Series}
        \textbf{Most time series models we study are linear Gaussian!}
        \begin{itemize}
            \item AR, MA, ARMA processes with Gaussian innovations
            \item VARs and linearized DSGE models
            \item Wold decomposition assumes covariance stationarity
        \end{itemize}
        
        \vspace{0.3em}
        
        $\Rightarrow$ \textbf{Second-moment properties} (autocovariances) become the key object for identification!
    \end{block}
    
    \vspace{0.3em}
    
    \textbf{Implication}: If two parameter values yield the same autocovariance structure, they are \textbf{observationally equivalent}.
\end{frame}

% ----------------------------------------------------------------------------
\begin{frame}{Observational Equivalence}
    \begin{definition}[Observational Equivalence]
        Two models $\mathcal{M}(\theta_1)$ and $\mathcal{M}(\theta_2)$ are \textbf{observationally equivalent} if:
        \[
        \mathbb{D}(\mathbf{Y}^T \mid \theta_1) = \mathbb{D}(\mathbf{Y}^T \mid \theta_2)
        \]
    \end{definition}
    
    \vspace{0.5em}
    
    \textbf{In Time Series}: For covariance-stationary Gaussian processes, observational equivalence reduces to having the \textbf{same second-moment properties}.
    
    \vspace{0.5em}
    
    \begin{alertblock}{The Identification Challenge}
        Different structural models/parameters can generate \textbf{identical autocovariance functions}.
        
        \vspace{0.3em}
        
        $\Rightarrow$ Data alone cannot distinguish between them!
    \end{alertblock}
    
    \vspace{0.3em}
    
    \textbf{Next}: How do we formalize ``same second-moment properties''?
\end{frame}

% ----------------------------------------------------------------------------
\begin{frame}{The Autocovariance Generating Function (ACGF)}

    \vspace{-0.5em}

    \textbf{Tool for comparing second-moment properties}:
    
    \begin{definition}[ACGF]
        For a covariance-stationary process with autocovariances $\Gamma(j)$:
        \[
        \Omega(z) = \sum_{j=-\infty}^{\infty} \Gamma(j) z^j
        \]
    \end{definition}
    
    % \vspace{0.5em}
    
    % The ACGF is a z-transform that converts the infinite sequence of autocovariances into a single analytic function.
    
    % \vspace{0.5em}
    
    \begin{block}{Key Insight: The Identification Challenge in Macroeconometrics}
        \textbf{Two time series models are observationally equivalent} if and only if they have the \textbf{same ACGF}:
        \[
        \Omega(z;\theta_1) = \Omega(z;\theta_2) \quad \forall z
        \]
        
        This is the root of why structural interpretation requires \textbf{additional identifying assumptions}!
    \end{block}
\end{frame}

% ----------------------------------------------------------------------------
\begin{frame}{Preview: Types of Indeterminacy}
    Observational equivalence can lead to \textbf{indeterminacy} --- multiple structural interpretations consistent with the same data.
    
    \vspace{0.5em}
    
    \textbf{Three main sources} (covered in detail in the VAR section):
    \begin{enumerate}
        \item \textbf{Static Indeterminacy}: Orthogonal rotations of shocks
        \item \textbf{Dynamic Indeterminacy}: Non-fundamental representations
        \item \textbf{Size Indeterminacy}: More shocks than observables
    \end{enumerate}
    
    \vspace{0.5em}
    
    \begin{block}{Example: MA(1) ``Root Flipping''}
        Consider $y_t = u_t + \theta u_{t-1}$ with $|\theta| < 1$.
        
        The process with $\theta^* = 1/\theta$ and rescaled variance has \textbf{identical autocovariances}!
        
        \vspace{0.2em}
        
        Both: $\rho_1 = \frac{\theta}{1+\theta^2}$ (same for $\theta$ and $1/\theta$)
    \end{block}
    
    \vspace{0.3em}
    
    \textit{We'll develop identification strategies when we cover Structural VARs.}
\end{frame}

% ----------------------------------------------------------------------------
\begin{frame}{Causality and Invertibility: General Definitions}
    Let $\{Y_t\}$ be a stationary process and $\{Z_t\}$ an input/shock process. Define filtrations:
    \[
    \mathcal{F}^Z_t = \sigma(Z_t, Z_{t-1}, \ldots), \qquad \mathcal{F}^Y_t = \sigma(Y_t, Y_{t-1}, \ldots)
    \]
    
    \begin{definition}[Causality]
        $\{Y_t\}$ is \textbf{causal with respect to} $\{Z_t\}$ if $Y_t$ is $\mathcal{F}^Z_t$-measurable for all $t$.
    \end{definition}
    
    \textbf{Interpretation}: $Y_t$ depends only on \textbf{current and past} values of $Z$ --- no future inputs needed.
    
    \vspace{0.3em}
    
    \begin{definition}[Invertibility]
        $\{Y_t\}$ is \textbf{invertible with respect to} $\{Z_t\}$ if $Z_t$ is $\mathcal{F}^Y_t$-measurable for all $t$.
    \end{definition}
    
    \textbf{Interpretation}: The shocks $Z_t$ can be \textbf{recovered from current and past observables} $Y_t$.
    
    \vspace{0.3em}
    
    \begin{alertblock}{Important Caveat}
        These are \textbf{not intrinsic properties} of a stationary process --- you must specify the representation (input/shock process). Stationarity guarantees a causal Wold MA($\infty$) w.r.t.\ Wold innovations, but does \textbf{not} imply a given structural shock is recoverable.
    \end{alertblock}
\end{frame}

% ----------------------------------------------------------------------------
\begin{frame}{VARMA Case: Sufficient Conditions}
    Consider a VARMA process: $\Phi(L) Y_t = \Theta(L) Z_t$, where $Z_t \sim \text{WN}(0, \Sigma)$.
    
    \vspace{0.3em}
    
    \begin{block}{Sufficient Condition for Causality}
        $\{Y_t\}$ is \textbf{causal} w.r.t.\ $\{Z_t\}$ if $\det \Phi(z) \neq 0$ for all $|z| \leq 1$.
        
        \vspace{0.2em}
        
        $\Rightarrow$ Roots of $\det \Phi(z) = 0$ lie \textbf{outside the unit circle}.
        
        \vspace{0.2em}
        
        This yields the VMA($\infty$) representation: $Y_t = \Phi(L)^{-1}\Theta(L) Z_t = \Psi(L) Z_t$ with $\Psi(L)$ one-sided.
    \end{block}
    
    \vspace{0.3em}
    
    \begin{block}{Sufficient Condition for Invertibility}
        $\{Y_t\}$ is \textbf{invertible} w.r.t.\ $\{Z_t\}$ if $\det \Theta(z) \neq 0$ for all $|z| \leq 1$.
        
        \vspace{0.2em}
        
        $\Rightarrow$ Roots of $\det \Theta(z) = 0$ lie \textbf{outside the unit circle}.
        
        \vspace{0.2em}
        
        This yields the VAR($\infty$) representation: $\Pi(L) Y_t = Z_t$ with $\Pi(L)$ one-sided.
    \end{block}
    
    \vspace{0.3em}
    
    \begin{alertblock}{Warning (Wolf, 2022)}
        Invertibility is \textbf{far from guaranteed} in structural models! E.g., 5 shocks but only 2 observables $\Rightarrow$ cannot possibly be invertible.
    \end{alertblock}
\end{frame}

% ----------------------------------------------------------------------------
\begin{frame}{Why Invertibility Matters}
    \textbf{For estimation and structural analysis}:
    
    \vspace{0.5em}
    
    If the process is invertible:
    \begin{itemize}
        \item Innovations $z_t$ can be recovered from data
        \item We can estimate VAR and back out the shocks
        \item IRFs and variance decompositions become meaningful
    \end{itemize}
    
    \vspace{0.5em}
    
    \textbf{If NOT invertible}:
    \begin{itemize}
        \item VAR residuals are \textbf{not} the structural shocks
        \item Standard SVAR methods may give misleading results
        \item Need alternative identification strategies
    \end{itemize}
    
    \vspace{0.5em}
    
    \begin{block}{Necessary Condition}
        For invertibility: $n_{\text{shocks}} \leq n_{\text{observables}}$
        
        But this is \textbf{not sufficient}! (Forward-looking behavior can break invertibility)
    \end{block}
\end{frame}

% ----------------------------------------------------------------------------
\begin{frame}{Part I: Summary}
    \begin{center}
    \small
    \begin{tabular}{p{3.8cm}p{8.5cm}}
        \toprule
        \textbf{Concept} & \textbf{Key Point} \\
        \midrule
        Global Identification & Unique $\theta$ in entire parameter space \\[0.4em]
        Local Identification & Unique in neighborhood \\[0.4em]
        Linear Gaussian & Second moments fully characterize distribution \\[0.4em]
        Observational Equiv. & Same ACGF $\Rightarrow$ indistinguishable \\[0.4em]
        Causality & $Y_t$ is $\mathcal{F}^Z_t$-measurable (depends on current/past $Z$'s) \\[0.4em]
        Invertibility & $Z_t$ is $\mathcal{F}^Y_t$-measurable (recoverable from current/past $Y$'s) \\
        \bottomrule
    \end{tabular}
    \end{center}
    
    \vspace{0.5em}
    
    \textbf{Key Insight}: In linear Gaussian time series, identification hinges on second-moment properties. Observational equivalence $\Rightarrow$ \textbf{indeterminacy}, which requires additional assumptions to resolve (covered in VAR section).
\end{frame}

% ============================================================================
% PART II: REDUCED-FORM MODELS
% ============================================================================
\begin{frame}[plain]
    \begin{center}
        {\Large \textbf{Part II: Reduced-Form Models}}\\[1em]
        {\large AR, MA, and ARMA Processes}
    \end{center}
\end{frame}

% ----------------------------------------------------------------------------
\begin{frame}{The AR($p$) Process: Definition}
    \begin{definition}[Autoregressive Process of Order $p$]
        A stochastic process $\{Y_t\}$ is AR($p$) if:
        \[
        Y_t = \E[Y_t \mid \F_{t-1}] + u_t = \phi_0 + \phi_1 y_{t-1} + \cdots + \phi_p y_{t-p} + u_t
        \]
        Equivalently: $\Phi_p(L)y_t = \phi_0 + u_t$, where:
        \[
        \Phi_p(L) = 1 - \phi_1 L - \phi_2 L^2 - \cdots - \phi_p L^p
        \]
        and $u_t$ is white noise: $\E[u_t \mid \F_{t-1}] = 0$, $\E[u_t^2 \mid \F_{t-1}] = \sigma^2$.
    \end{definition}
    
    \vspace{0.5em}
    
    \textbf{Interpretation}: Current value = conditional expectation given past + innovation.
    
    \vspace{0.3em}
    
    \textbf{Note}: AR processes are \textbf{always invertible} (by construction, $u_t = y_t - \E[y_t \mid \F_{t-1}]$).
\end{frame}

% ----------------------------------------------------------------------------
\begin{frame}{The Companion Form}
    \textbf{Idea}: Cast AR($p$) as a VAR(1) by stacking variables.
    
    \begin{definition}[Companion Form]
        \[
        \underbrace{\begin{bmatrix} Y_t \\ Y_{t-1} \\ \vdots \\ Y_{t-p+1} \end{bmatrix}}_{\mathbf{X}_t} = 
        \begin{bmatrix} \phi_0 \\ 0 \\ \vdots \\ 0 \end{bmatrix} +
        \underbrace{\begin{bmatrix} \phi_1 & \phi_2 & \cdots & \phi_p \\ 1 & 0 & \cdots & 0 \\ \vdots & \ddots & & \vdots \\ 0 & \cdots & 1 & 0 \end{bmatrix}}_{F}
        \underbrace{\begin{bmatrix} Y_{t-1} \\ Y_{t-2} \\ \vdots \\ Y_{t-p} \end{bmatrix}}_{\mathbf{X}_{t-1}} +
        \begin{bmatrix} u_t \\ 0 \\ \vdots \\ 0 \end{bmatrix}
        \]
    \end{definition}
    
    Compactly: $\mathbf{X}_t = c + F\mathbf{X}_{t-1} + v_t$
    
    \vspace{0.5em}
    
    \textbf{Why useful?}
    \begin{itemize}
        \item $\{\mathbf{X}_t\}$ is Markovian even though $\{Y_t\}$ is not
        \item Stationarity analysis reduces to eigenvalue analysis of $F$
    \end{itemize}
\end{frame}

% ----------------------------------------------------------------------------
\begin{frame}{AR($p$): Stationarity Conditions}
    \begin{theorem}[Asymptotic Stationarity of AR($p$)]
        An AR($p$) process is \textbf{asymptotically stationary} if the eigenvalues of the companion matrix $F$ are all \textbf{inside the unit circle}.
    \end{theorem}
    
    \vspace{0.3em}
    
    \begin{proof}[Sketch]
        Iterating backward: $\mathbf{X}_t = F^t \mathbf{X}_0 + \sum_{i=0}^{t-1} F^i v_{t-i}$. For this not to explode, need $F^t \to 0$, which requires $|\lambda_i| < 1$ for all eigenvalues.
    \end{proof}
    
    \vspace{0.5em}
    
    \begin{corollary}[Characteristic Polynomial Version]
        AR($p$) is stationary iff roots of $\Phi_p(z) = 1 - \phi_1 z - \cdots - \phi_p z^p = 0$ lie \textbf{outside the unit circle}.
    \end{corollary}
    
    \textbf{Connection}: Eigenvalues of $F$ are reciprocals of polynomial roots.
\end{frame}

% ----------------------------------------------------------------------------
\begin{frame}{AR($p$): Moments (Yule-Walker Equations)}
    For a weakly-stationary AR($p$) process:
    
    \vspace{0.5em}
    
    \textbf{Expected value}:
    \[
    \E[y_t] = \frac{\phi_0}{1 - \sum_{i=1}^p \phi_i}
    \]
    
    % \vspace{0.3em}
    
    \textbf{Autocovariance} (Yule-Walker Equations):
    \[
    \gamma_k = \sum_{i=1}^p \phi_i \gamma_{k-i}, \quad k \geq 1
    \]
    
    % \vspace{0.3em}
    
    \textbf{Variance}:
    \[
    \gamma_0 = \sigma^2 + \sum_{i=1}^p \phi_i \gamma_i
    \]
    
    % \vspace{0.5em}
    
    \begin{block}{Key Pattern}
        ACF \textbf{decays geometrically} --- signature of AR processes!
    \end{block}
\end{frame}

% ----------------------------------------------------------------------------
\begin{frame}{AR(1) Example: Monetary Policy and Inflation}
    \textbf{Recall from Lecture 1}: The simple monetary transmission model:
    \begin{align*}
        y_t &= -\gamma(i_t - \E_t[\pi_{t+1}]) + u_{1t} \quad \text{(IS curve)} \\
        \pi_t &= \lambda y_t + u_{2t} \quad \text{(Phillips curve)} \\
        i_t &= \rho\pi_t + i^* \quad \text{(Taylor rule)}
    \end{align*}
    
    \vspace{0.3em}
    
    \textbf{Reduced form for inflation}:
    \[
    \pi_t = \phi_0 + \phi_1 \pi_{t-1} + v_t
    \]
    where $\phi_1 = 1 + \lambda\gamma(\rho - 1)$ (function of structural parameters!)
    
    \vspace{0.3em}
    
    \textbf{Stability requires}: $|\phi_1| < 1$ $\Leftrightarrow$ \textbf{Taylor Principle}: $\rho > 1$
    
    \vspace{0.3em}
    
    \textit{Key insight}: The reduced-form AR coefficient encodes structural policy parameters!
\end{frame}

% ----------------------------------------------------------------------------
\begin{frame}{AR(1) Inflation Example: Behavior by $\rho$}
    \textbf{Cases} (with $\lambda = 0.5$, $\gamma = -0.5$):
    
    \vspace{0.5em}
    
    \begin{center}
    \begin{tabular}{lll}
        \toprule
        \textbf{Taylor Rule $\rho$} & \textbf{Reduced-form $\phi_1$} & \textbf{Behavior} \\
        \midrule
        $\rho = 0$ & $\phi_1 > 1$ & Explosive inflation \\
        $0 < \rho < 1$ & $\phi_1 > 1$ & Explosive inflation \\
        $\rho = 1$ & $\phi_1 = 1$ & Random walk \\
        $1 < \rho < 5$ & $0 < \phi_1 < 1$ & Stationary, persistent \\
        $5 < \rho < 9$ & $-1 < \phi_1 < 0$ & Stationary, oscillating \\
        $\rho > 9$ & $\phi_1 < -1$ & Explosive (over-reaction) \\
        \bottomrule
    \end{tabular}
    \end{center}
    
    \vspace{0.5em}
    
    \textbf{ACF}: $\rho_k = \phi_1^k$ --- geometric decay toward zero.
    
    \vspace{0.3em}
    
    See figures in Appendix \hyperlink{slide:ar1_figures}{\beamerbutton{$\rightarrow$ Figures}}
\end{frame}

% ----------------------------------------------------------------------------
\begin{frame}{The MA($q$) Process: Definition}
    \begin{definition}[Moving Average Process of Order $q$]
        A process $\{y_t\}$ is MA($q$) if:
        \[
        y_t = \theta_0 + u_t + \theta_1 u_{t-1} + \cdots + \theta_q u_{t-q} = \theta_0 + \theta_q(L) u_t
        \]
        where $\theta_q(L) = 1 + \theta_1 L + \cdots + \theta_q L^q$ and $u_t \sim WN(0, \sigma^2)$.
    \end{definition}
    
    \vspace{0.5em}
    
    \textbf{Key difference from AR}: Current value is a weighted sum of current and \textbf{past innovations} (not past values of $y$).
    
    \vspace{0.5em}
    
    \begin{proposition}
        All finite-order MA($q$) processes are \textbf{covariance-stationary} and \textbf{mean-ergodic}.
    \end{proposition}
    
    \textbf{Intuition}: Finite sum of stationary (white noise) terms is always stationary.
\end{frame}

% ----------------------------------------------------------------------------
\begin{frame}{MA($q$): Moments}
    \textbf{Expected value}: $\E[y_t] = \theta_0$
    
    \vspace{0.3em}
    
    \textbf{Variance}:
    \[
    \gamma_0 = \sigma^2\left(1 + \sum_{j=1}^q \theta_j^2\right)
    \]
    
    \vspace{0.3em}
    
    \textbf{Autocovariance}:
    \[
    \gamma_k = \begin{cases}
        \left(\theta_k + \sum_{j=1}^{q-k} \theta_j \theta_{j+k}\right)\sigma^2 & \text{if } k \leq q \\
        0 & \text{if } k > q
    \end{cases}
    \]
    
    \vspace{0.5em}
    
    \begin{block}{Key Pattern}
        ACF \textbf{cuts off} after lag $q$ --- signature of MA processes!
        
        (Contrast with AR: geometric decay)
    \end{block}
\end{frame}

% ----------------------------------------------------------------------------
\begin{frame}{MA($q$): Invertibility}
    \begin{definition}[Invertibility]
        An MA($q$) process is \textbf{invertible} if $u_t$ lies in the space spanned by current and past observables: $u_t \in \Span(y_\tau, -\infty < \tau \leq t)$.
    \end{definition}
    
    \vspace{0.3em}
    
    Equivalently, MA($q$) can be written as AR($\infty$):
    \[
    u_t = \theta_q(L)^{-1}(y_t - \theta_0) = \sum_{j=0}^{\infty} \psi_j(y_{t-j} - \theta_0)
    \]
    
    \vspace{0.5em}
    
    \begin{theorem}[Invertibility Condition]
        MA($q$) is invertible iff roots of $\theta_q(z) = 1 + \theta_1 z + \cdots + \theta_q z^q = 0$ lie \textbf{outside the unit circle}.
    \end{theorem}
    
    \vspace{0.3em}
    
    \textbf{Practical importance}: Invertibility allows expressing $u_t$ as function of observables --- essential for estimation!
\end{frame}

% ----------------------------------------------------------------------------
\begin{frame}{MA(1): Observational Equivalence Example}
    \textbf{Key fact}: For any MA process, there exists a family of observationally equivalent alternatives with the same mean and autocovariances.
    
    \vspace{0.5em}
    
    \textbf{Example}: MA(1) with $y_t = u_t + \theta u_{t-1}$, $u_t \sim (0,\sigma^2)$.
    
    \vspace{0.3em}
    
    \textbf{Autocorrelation at lag 1}:
    \[
    \rho_1 = \frac{\theta}{1+\theta^2}
    \]
    
    \textbf{Observation}: Both $\theta$ and $1/\theta$ give the \textbf{same} $\rho_1$!
    
    \vspace{0.3em}
    
    The \textbf{invertible} representation ($|\theta| < 1$) is obs. equivalent to a \textbf{non-invertible} one:
    \begin{itemize}
        \item Original: $\theta$, $\sigma^2$
        \item Alternative: $\theta^* = 1/\theta$, $\sigma^{*2} = \theta^2 \sigma^2$
    \end{itemize}
    
    \vspace{0.3em}
    
    \begin{alertblock}{Resolution}
        We \textbf{cannot test} statistically whether a process is invertible!
        
        We impose invertibility as an \textit{a priori} identifying assumption.
    \end{alertblock}
\end{frame}

% ----------------------------------------------------------------------------
\begin{frame}{MA Observational Equivalence: ACGF Perspective}
    \textbf{Tool}: The Autocovariance Generating Function.
    
    \begin{proposition}
        For MA($q$): $y_t - \theta_0 = \theta_q(L)\varepsilon_t$, the ACGF is:
        \[
        \Gamma_Y(z) = \theta_q(z)\theta_q(z^{-1})\sigma^2_\varepsilon
        \]
    \end{proposition}
    
    \vspace{0.5em}
    
    \textbf{Two MA processes are observationally equivalent} if they have:
    \begin{enumerate}
        \item Same mean: $\theta_0^X = \theta_0^Y$
        \item Same ACGF: $\Gamma_X(z) = \Gamma_Y(z)$
    \end{enumerate}
    
    \vspace{0.5em}
    
    \textbf{For MA(1)}: Matching $\Gamma(z) = (1+\theta z)(1+\theta z^{-1})\sigma^2$ allows the ``flip'':
    \[
    \theta \leftrightarrow 1/\theta, \quad \sigma^2 \leftrightarrow \theta^2\sigma^2
    \]
\end{frame}

% ----------------------------------------------------------------------------
\begin{frame}{The ARMA($p,q$) Process}

    \vspace{-1em}

    \begin{definition}[ARMA Process]
        A process $\{Y_t\}$ is ARMA($p,q$) if:
        \[
        Y_t = \phi_0 + \sum_{i=1}^p \phi_i Y_{t-i} + u_t + \sum_{j=1}^q \theta_j u_{t-j}
        \]
        Or: $\Phi_p(L)Y_t = \phi_0 + \Theta_q(L)u_t$
    \end{definition}
    
    % \vspace{0.5em}
    
    \begin{proposition}[Stationarity and Invertibility]
        \begin{itemize}
            \item \textbf{Stationarity}: Roots of $\Phi_p(z) = 0$ outside unit circle (AR part)
            \item \textbf{Invertibility}: Roots of $\Theta_q(z) = 0$ outside unit circle (MA part)
        \end{itemize}
    \end{proposition}
    
    % \vspace{0.3em}
    
    \textbf{ACGF}:
    \[
    \Gamma(z) = \frac{\Theta_q(z)\Theta_q(z^{-1})}{\Phi_p(z)\Phi_p(z^{-1})}\sigma^2_\varepsilon
    \]
\end{frame}

% ----------------------------------------------------------------------------
\begin{frame}{ARMA: Common Factor Problem}
    \textbf{Identification issue}: If AR and MA polynomials share a common root, the model is \textbf{not identified}.
    
    \vspace{0.5em}
    
    \textbf{Example}: ARMA(1,1) with $(1-\phi L)y_t = (1+\theta L)u_t$
    
    If $\phi = -\theta$ (common factor):
    \[
    (1-\phi L)y_t = (1-\phi L)u_t \implies y_t = u_t
    \]
    
    The ARMA(1,1) is \textbf{observationally equivalent to white noise}!
    
    \vspace{0.5em}
    
    \begin{alertblock}{Solution}
        For a \textbf{canonical} (identified) ARMA representation, require:
        
        $\Phi_p(z)$ and $\Theta_q(z)$ have \textbf{no common roots}.
    \end{alertblock}
\end{frame}

% ----------------------------------------------------------------------------
\begin{frame}{ARMA: Equivalent Representations}
    A stationary and invertible ARMA($p,q$) admits:
    
    \vspace{0.5em}
    
    \textbf{1. MA($\infty$) Representation}:
    \[
    y_t = \mu + \psi(L)u_t = \mu + \sum_{j=0}^{\infty} \psi_j u_{t-j}
    \]
    where $\psi(L) = \Phi_p(L)^{-1}\Theta_q(L)$
    
    \vspace{0.5em}
    
    \textbf{2. AR($\infty$) Representation}:
    \[
    \pi(L)(y_t - \mu) = u_t
    \]
    where $\pi(L) = \Theta_q(L)^{-1}\Phi_p(L)$
    
    \vspace{0.5em}
    
    \begin{block}{Key Insight}
        ARMA provides a \textbf{parsimonious parametric approximation} to MA($\infty$)/AR($\infty$) --- crucial for Wold decomposition!
    \end{block}
\end{frame}

% ----------------------------------------------------------------------------
\begin{frame}{Summary: AR vs MA vs ARMA}
    \begin{center}
    \small
    \begin{tabular}{lccc}
        \toprule
        \textbf{Property} & \textbf{AR($p$)} & \textbf{MA($q$)} & \textbf{ARMA($p,q$)} \\
        \midrule
        Stationarity & Roots of $\Phi(z)$ & Always & Roots of $\Phi(z)$ \\
         & outside UC & stationary & outside UC \\[0.5em]
        Invertibility & Always & Roots of $\Theta(z)$ & Roots of $\Theta(z)$ \\
         & invertible & outside UC & outside UC \\[0.5em]
        ACF pattern & Geometric & Cuts off & Geometric \\
         & decay & after lag $q$ & decay \\[0.5em]
        PACF pattern & Cuts off & Geometric & Geometric \\
         & after lag $p$ & decay & decay \\
        \bottomrule
    \end{tabular}
    \end{center}
    
    \vspace{0.3em}
    
    \textbf{UC} = Unit Circle; \textbf{PACF} = Partial Autocorrelation Function
\end{frame}

% ============================================================================
% PART III: WOLD DECOMPOSITION
% ============================================================================
\begin{frame}[plain]
    \begin{center}
        {\Large \textbf{Part III: Wold's Decomposition Theorem}}\\[1em]
        {\large The Fundamental Representation of Stationary Processes}
    \end{center}
\end{frame}

% ----------------------------------------------------------------------------
\begin{frame}{Wold's Representation Theorem}
    \begin{theorem}[Wold, 1938]
        Let $\{y_t\}$ be an $n$-dimensional \textbf{covariance-stationary} time series. There exists an $n \times n$ lag polynomial $\Psi(L)$ and $u_t \sim WN(0, \Sigma)$ such that:
        \[
        y_t = \Psi(L) u_t + d_t, \quad \Psi(L) = I + \sum_{\ell=1}^{\infty} \Psi_\ell L^\ell
        \]
        where:
        \begin{enumerate}
            \item $\Psi(L)$ is \textbf{square-summable}
            \item $u_t = y_t - \E^*[y_t \mid \{y_\tau\}_{-\infty < \tau \leq t-1}]$ (one-step-ahead forecast errors)
            \item $\{y_t\}$ is \textbf{invertible} with respect to $u_t$
            \item $\{d_t\}$ is a \textbf{purely deterministic} process
        \end{enumerate}
    \end{theorem}
    
    \vspace{0.3em}
    
    \textbf{In words}: Any covariance-stationary time series = VMA($\infty$) + deterministic
\end{frame}

% ----------------------------------------------------------------------------
\begin{frame}{Why Wold Matters: The Key Insight}
    % \textbf{From Wolf (2022)}:
    
    % \vspace{0.5em}
    
    The Wold decomposition does something \textbf{very simple}: it splits a process $\{y_t\}$ into:
    \begin{enumerate}
        \item \textbf{One-step-ahead prediction errors} $u_t$
        \item \textbf{A perfectly predictable residual} $d_t$
    \end{enumerate}
    
    \vspace{0.5em}
    
    \begin{block}{Three Equivalent Representations}
        For second-order properties, we can \textbf{freely map} between:
        \begin{enumerate}
            \item \textbf{Autocovariance function}: $\{\Gamma_k\}_{k=0}^{\infty}$
            \item \textbf{Spectral density}: $s_y(\omega)$
            \item \textbf{Wold decomposition}: $\{\Psi_j, \Sigma\}$
        \end{enumerate}
        
        \vspace{0.2em}
        
        \textbf{Mapping}: $\Psi_\ell = \Cov(y_t, u_{t-\ell})\Sigma^{-1}$
    \end{block}
    
    \vspace{0.3em}
    
    \textbf{Key}: The Wold decomposition is \textbf{identifiable from data}!
\end{frame}

% ----------------------------------------------------------------------------
\begin{frame}{Wold: Practical Implications}
    \textbf{Problem}: MA($\infty$) has infinitely many parameters --- cannot estimate!
    
    \vspace{0.5em}
    
    \textbf{Solution}: ARMA($p,q$) provides a parsimonious approximation:
    \[
    \Phi_p(L)y_t = \Theta_q(L)u_t \quad \Leftrightarrow \quad y_t = \frac{\Theta_q(L)}{\Phi_p(L)}u_t
    \]
    
    The ratio $\psi(L) = \Theta_q(L)/\Phi_p(L)$ approximates the Wold coefficients $\{\Psi_i\}$.
    
    \vspace{0.5em}
    
    \begin{block}{Why This Matters}
        \begin{itemize}
            \item We can \textbf{estimate} finite-parameter ARMA/VAR models
            \item These approximate the infinite-parameter Wold representation
            \item From VAR, we can also get VAR($\infty$) representation: $A(L)y_t = u_t$
        \end{itemize}
    \end{block}
\end{frame}

% ----------------------------------------------------------------------------
\begin{frame}{Wold vs. Structural: A Critical Warning}
    \begin{alertblock}{Wold Innovations $\neq$ Structural Shocks!}
        The Wold coefficients $\{\Psi_j\}$ are \textbf{reduced-form} objects.
        
        \textbf{Nothing guarantees they are interesting} --- they are just coefficients on reduced-form prediction errors!
    \end{alertblock}
    
    % \vspace{0.5em}
    
    \textbf{Structural VMA} (what we want):
    \[
    y_t = \sum_{\ell=0}^{\infty} \Theta_\ell \varepsilon_{t-\ell}, \quad \varepsilon_t = \text{structural shocks}
    \]
    
    \textbf{Wold representation} (what we estimate):
    \[
    y_t = \sum_{j=0}^{\infty} \Psi_j u_{t-j}, \quad u_t = \text{prediction errors}
    \]
    
    \vspace{0.3em}
    
    \textbf{The gap}: Identifying $\Theta_\ell$ from $\Psi_j$ requires \textbf{additional assumptions}!
    
    $\Rightarrow$ This is the \textbf{identification problem} we'll address in Structural VARs.
\end{frame}

% ----------------------------------------------------------------------------
\begin{frame}{Summary}
    \textbf{Identification}:
    \begin{itemize}
        \item Global vs. local: uniqueness in entire space vs. neighborhood
        \item Linear Gaussian $\Rightarrow$ second moments matter
        \item Observational equivalence: same ACGF $\Rightarrow$ indistinguishable
        \item Causality: $y_t$ from current/past $z$'s; Invertibility: $z_t$ from $y$'s
    \end{itemize}
    
    \vspace{0.3em}
    
    \textbf{Reduced-Form Models}:
    \begin{itemize}
        \item AR($p$): Stationary if roots outside UC; ACF decays geometrically
        \item MA($q$): Always stationary; invertible if roots outside UC; ACF cuts off
        \item ARMA: Combines both; common factor $\Rightarrow$ non-identification
    \end{itemize}
    
    \vspace{0.3em}
    
    \textbf{Wold's Theorem}:
    \begin{itemize}
        \item Any stationary process = VMA($\infty$) + deterministic
        \item Wold innovations = prediction errors (\textbf{not} structural shocks!)
        \item Reduced-form, identifiable; structural interpretation needs more
    \end{itemize}
\end{frame}

% ============================================================================
% PART IV: EXERCISES
% ============================================================================
\begin{frame}[plain]
    \begin{center}
        {\Large \textbf{Part IV: Exercises}}\\[1em]
        % {\large Hour 2}
    \end{center}
\end{frame}

% ----------------------------------------------------------------------------
% TIM'S EXERCISES
% ----------------------------------------------------------------------------
\begin{frame}{Tim's Practice Problems: Exercise 2}
    \begin{block}{Problem Statement}
        \textit{[Exercise 2 from practice\_problems\_week2.pdf]}
        
        \vspace{0.5em}
        
        % PLACEHOLDER: Insert the actual problem statement here
        Problem to be inserted from the problem set.
    \end{block}
    
    \vspace{1em}
    
    \textbf{Solution Approach}:
    \begin{enumerate}
        \item Step 1: [To be completed]
        \item Step 2: [To be completed]
        \item Step 3: [To be completed]
    \end{enumerate}
\end{frame}

\begin{frame}{Exercise 2: Solution}
    \textbf{Detailed Solution}:
    
    \vspace{1em}
    
    % PLACEHOLDER: Insert detailed solution here
    [Solution steps to be added once problem statement is available]
    
    \vspace{1em}
    
    \begin{block}{Key Takeaway}
        [Main insight from this exercise]
    \end{block}
\end{frame}

% ----------------------------------------------------------------------------
\begin{frame}{Tim's Practice Problems: Exercise 3}
    \begin{block}{Problem Statement}
        \textit{[Exercise 3 from practice\_problems\_week2.pdf]}
        
        \vspace{0.5em}
        
        % PLACEHOLDER: Insert the actual problem statement here
        Problem to be inserted from the problem set.
    \end{block}
    
    \vspace{1em}
    
    \textbf{Solution Approach}:
    \begin{enumerate}
        \item Step 1: [To be completed]
        \item Step 2: [To be completed]
        \item Step 3: [To be completed]
    \end{enumerate}
\end{frame}

\begin{frame}{Exercise 3: Solution}
    \textbf{Detailed Solution}:
    
    \vspace{1em}
    
    % PLACEHOLDER: Insert detailed solution here
    [Solution steps to be added once problem statement is available]
    
    \vspace{1em}
    
    \begin{block}{Key Takeaway}
        [Main insight from this exercise]
    \end{block}
\end{frame}

% ----------------------------------------------------------------------------
% MEDEIROS EXERCISES - ARMA
% ----------------------------------------------------------------------------
\begin{frame}{Medeiros ARMA: Exercise 1 --- Statement}
    \begin{block}{Problem}
        Let $y_t$, $t = 1, \ldots, T$, be a time series described by an MA(1) process:
        \[
        y_t = \mu + \varepsilon_t + \theta \varepsilon_{t-1}, \quad \varepsilon_t \sim \text{IID}(0, \sigma_\varepsilon^2)
        \]
        Suppose that $y_t$ is observed with \textbf{additive measurement noise}, i.e., $x_t = y_t + u_t$, where $u_t \sim \text{IID}(0, \sigma_u^2)$ and $\E(\varepsilon_t u_s) = 0$ for all $t, s$.
    \end{block}
    
    \vspace{0.5em}
    
    \textbf{Compute}:
    \begin{enumerate}[(a)]
        \item The mean and variance of $x_t$.
        \item The autocovariances of $x_t$.
        \item Which ARMA$(p, q)$ process best describes $x_t$?
    \end{enumerate}
\end{frame}

\begin{frame}{Medeiros ARMA: Exercise 1 --- Solution (a)}
    \textbf{(a) Mean and Variance of $x_t$}:
    
    \vspace{0.5em}
    
    Since $x_t = y_t + u_t = \mu + \varepsilon_t + \theta \varepsilon_{t-1} + u_t$:
    
    \vspace{0.3em}
    
    \textbf{Mean}:
    \[
    \E(x_t) = \E(\mu + \varepsilon_t + \theta \varepsilon_{t-1} + u_t) = \mu
    \]
    
    \vspace{0.3em}
    
    \textbf{Variance}: Since $\varepsilon_t$, $\varepsilon_{t-1}$, and $u_t$ are mutually uncorrelated:
    \begin{align*}
    \gamma_0 &= \Var(x_t) = \Var(\varepsilon_t) + \theta^2 \Var(\varepsilon_{t-1}) + \Var(u_t) \\
    &= \sigma_\varepsilon^2 + \theta^2 \sigma_\varepsilon^2 + \sigma_u^2 = \boxed{(1 + \theta^2)\sigma_\varepsilon^2 + \sigma_u^2}
    \end{align*}
\end{frame}

\begin{frame}{Medeiros ARMA: Exercise 1 --- Solution (b)}
    \textbf{(b) Autocovariances of $x_t$}:
    
    \vspace{0.5em}
    
    For $k \geq 1$:
    \[
    \gamma_k = \Cov(x_t, x_{t-k}) = \Cov(\varepsilon_t + \theta\varepsilon_{t-1} + u_t, \varepsilon_{t-k} + \theta\varepsilon_{t-k-1} + u_{t-k})
    \]
    
    \vspace{0.3em}
    
    \textbf{For $k = 1$}: The only surviving term is $\Cov(\theta\varepsilon_{t-1}, \varepsilon_{t-1})$:
    \[
    \gamma_1 = \theta \cdot \sigma_\varepsilon^2 = \boxed{\theta \sigma_\varepsilon^2}
    \]
    
    \vspace{0.3em}
    
    \textbf{For $k \geq 2$}: There are no overlapping terms:
    \[
    \gamma_k = \boxed{0}, \quad k \geq 2
    \]
    
    \vspace{0.5em}
    
    \begin{block}{Result}
        The autocovariance structure is: $\gamma_0 = (1+\theta^2)\sigma_\varepsilon^2 + \sigma_u^2$, $\gamma_1 = \theta\sigma_\varepsilon^2$, $\gamma_k = 0$ for $k \geq 2$.
    \end{block}
\end{frame}

\begin{frame}{Medeiros ARMA: Exercise 1 --- Solution (c)}
    \textbf{(c) Which ARMA$(p,q)$ describes $x_t$?}
    
    \vspace{0.5em}
    
    The autocovariance ``cuts off'' after lag 1 $\Rightarrow$ \textbf{MA(1)} behavior!
    
    \vspace{0.3em}
    
    Therefore, $x_t$ can be written as:
    \[
    x_t = \mu + \eta_t + \tilde{\theta} \eta_{t-1}, \quad \eta_t \sim \text{WN}(0, \sigma_\eta^2)
    \]
    
    \vspace{0.3em}
    
    \textbf{Matching parameters}: We need:
    \[
    \gamma_0^{\text{MA}(1)} = (1 + \tilde{\theta}^2)\sigma_\eta^2 = (1+\theta^2)\sigma_\varepsilon^2 + \sigma_u^2
    \]
    \[
    \gamma_1^{\text{MA}(1)} = \tilde{\theta}\sigma_\eta^2 = \theta\sigma_\varepsilon^2
    \]
    
    \vspace{0.3em}
    
    \begin{alertblock}{Conclusion}
        $x_t$ is an \textbf{MA(1)} process, but with parameters different from the original MA(1)! Measurement noise ``hides'' the true structural parameters $(\theta, \sigma_\varepsilon^2)$.
    \end{alertblock}
\end{frame}

% ----------------------------------------------------------------------------
\begin{frame}{Medeiros ARMA: Exercise 2 --- Statement}
    \begin{block}{Problem}
        Let $\{y_t\}$ and $\{z_t\}$, $t = 1, \ldots, T$, be two stochastic processes:
        \[
        y_t = \phi_1 y_{t-1} + \varepsilon_t, \quad z_t = \phi_2 z_{t-1} + u_t
        \]
        where $\varepsilon_t \sim \text{IID}(0, \sigma_\varepsilon^2)$, $u_t \sim \text{IID}(0, \sigma_u^2)$, and $\E(\varepsilon_t u_s) = 0$ for all $t, s$.
    \end{block}
    
    \vspace{0.5em}
    
    \textbf{Show that}:
    \begin{enumerate}[(a)]
        \item If $\phi_1 = \phi_2$, then $x_t = y_t + z_t$ is an AR(1) process.
        \item If $\phi_1 \neq \phi_2$, then $x_t = y_t + z_t$ is an ARMA$(2, q)$ process with $q \leq 1$.
    \end{enumerate}
\end{frame}

\begin{frame}{Medeiros ARMA: Exercise 2 --- Solution (a)}
    \textbf{(a) Case $\phi_1 = \phi_2 = \phi$}:
    
    \vspace{0.5em}
    
    Adding the two equations:
    \begin{align*}
    y_t + z_t &= \phi(y_{t-1} + z_{t-1}) + \varepsilon_t + u_t \\
    x_t &= \phi x_{t-1} + w_t
    \end{align*}
    
    where $w_t = \varepsilon_t + u_t$.
    
    \vspace{0.5em}
    
    \textbf{Checking that $w_t$ is white noise}:
    \begin{itemize}
        \item $\E(w_t) = 0$
        \item $\Var(w_t) = \sigma_\varepsilon^2 + \sigma_u^2$
        \item $\Cov(w_t, w_{t-k}) = 0$ for $k \neq 0$ (since $\varepsilon$ and $u$ are serially independent and mutually uncorrelated)
    \end{itemize}
    
    \vspace{0.3em}
    
    \begin{block}{Conclusion}
        $x_t$ is an \textbf{AR(1)} process with parameter $\phi$ and innovation $w_t \sim \text{WN}(0, \sigma_\varepsilon^2 + \sigma_u^2)$.
    \end{block}
\end{frame}

\begin{frame}{Medeiros ARMA: Exercise 2 --- Solution (b) Part 1}
    \textbf{(b) Case $\phi_1 \neq \phi_2$}:
    
    \vspace{0.3em}
    
    \textbf{Strategy}: Write in lag-operator notation and combine.
    
    \vspace{0.3em}
    
    The individual processes are:
    \[
    (1 - \phi_1 L)y_t = \varepsilon_t, \quad (1 - \phi_2 L)z_t = u_t
    \]
    
    To eliminate $y_t$ and $z_t$ separately, apply cross-filters:
    \begin{align*}
    (1 - \phi_2 L)(1 - \phi_1 L)y_t &= (1 - \phi_2 L)\varepsilon_t \\
    (1 - \phi_1 L)(1 - \phi_2 L)z_t &= (1 - \phi_1 L)u_t
    \end{align*}
    
    \vspace{0.3em}
    
    Summing:
    \[
    (1 - \phi_1 L)(1 - \phi_2 L)(y_t + z_t) = (1 - \phi_2 L)\varepsilon_t + (1 - \phi_1 L)u_t
    \]
\end{frame}

\begin{frame}{Medeiros ARMA: Exercise 2 --- Solution (b) Part 2}
    \textbf{Expanding}:
    \[
    [1 - (\phi_1 + \phi_2)L + \phi_1\phi_2 L^2]x_t = \varepsilon_t - \phi_2\varepsilon_{t-1} + u_t - \phi_1 u_{t-1}
    \]
    
    \vspace{0.3em}
    
    \textbf{Right-hand side}: Define $\eta_t = \varepsilon_t + u_t - \phi_2\varepsilon_{t-1} - \phi_1 u_{t-1}$.
    
    \vspace{0.3em}
    
    Is this an MA(1) process in general? Compute autocovariances of $\eta_t$:
    \begin{itemize}
        \item $\gamma_0^\eta = (1 + \phi_2^2)\sigma_\varepsilon^2 + (1 + \phi_1^2)\sigma_u^2$
        \item $\gamma_1^\eta = -\phi_2 \sigma_\varepsilon^2 - \phi_1 \sigma_u^2$
        \item $\gamma_k^\eta = 0$ for $k \geq 2$
    \end{itemize}
    
    \vspace{0.3em}
    
    \begin{block}{Conclusion}
        $x_t$ follows an \textbf{ARMA(2,1)} process with:
        \begin{itemize}
            \item AR part: $(1 - \phi_1 L)(1 - \phi_2 L)$ with roots $1/\phi_1$ and $1/\phi_2$
            \item MA part: order $q \leq 1$ (it can be 0 if $\gamma_1^\eta = 0$)
        \end{itemize}
    \end{block}
\end{frame}

% ----------------------------------------------------------------------------
\begin{frame}{Medeiros ARMA: Exercise 3 --- Statement}
    \begin{block}{Problem}
        Consider two independent AR(1) processes $x_t$ and $y_t$ defined by:
        \[
        (1 - \rho_1 L)x_t = u_t, \; |\rho_1| < 1 \quad \text{and} \quad (1 - \rho_2 L)y_t = v_t, \; |\rho_2| < 1
        \]
        where $\E(u_t) = \E(v_t) = 0$, $\E(u_t v_\tau) = 0$ for all $t, \tau$,\\
        $\E(u_t u_\tau) = \sigma_u^2$ if $t = \tau$ (0 otherwise), $\E(v_t v_\tau) = \sigma_v^2$ if $t = \tau$.
        
        \vspace{0.3em}
        Let $z_t = x_t + y_t$.
    \end{block}
    
    \vspace{0.3em}
    
    \textbf{Tasks}:
    \begin{enumerate}[(a)]
        \item Compute $\E(x_{t+1}|\mathcal{F}_t)$, $\E(y_{t+1}|\mathcal{F}_t)$, $\E(z_{t+1}|\mathcal{F}_t)$, and the forecast error variance $V_D$.
        \item Show that $z_t$ is ARMA(2,1) and find $\theta$.
        \item Conditions for $z_t$ to be AR(2)? Show that if $\rho_1, \rho_2$ have the same sign, then it cannot be AR(2).
        \item Show that $\theta \neq \pm 1$.
    \end{enumerate}
\end{frame}

\begin{frame}{Medeiros ARMA: Exercise 3 --- Solution (a)}
    \textbf{(a) Conditional forecasts}:
    
    \vspace{0.3em}
    
    Since $x_t$ and $y_t$ are AR(1):
    \[
    \E(x_{t+1}|\mathcal{F}_t) = \rho_1 x_t, \quad \E(y_{t+1}|\mathcal{F}_t) = \rho_2 y_t
    \]
    
    Para $z_t = x_t + y_t$:
    \[
    \E(z_{t+1}|\mathcal{F}_t) = \rho_1 x_t + \rho_2 y_t
    \]
    
    \vspace{0.3em}
    
    \textbf{Forecast error} $e_{t+1} = z_{t+1} - \E(z_{t+1}|\mathcal{F}_t)$:
    \[
    e_{t+1} = (x_{t+1} + y_{t+1}) - (\rho_1 x_t + \rho_2 y_t) = u_{t+1} + v_{t+1}
    \]
    
    \vspace{0.3em}
    
    \textbf{Error variance} $V_D$:
    \[
    V_D = \Var(e_{t+1}) = \Var(u_{t+1}) + \Var(v_{t+1}) = \boxed{\sigma_u^2 + \sigma_v^2}
    \]
\end{frame}

\begin{frame}{Medeiros ARMA: Exercise 3 --- Solution (b) Part 1}
    \textbf{(b) Showing that $z_t$ is ARMA(2,1)}:
    
    \vspace{0.3em}
    
    From Exercise 2, we know that when $\rho_1 \neq \rho_2$:
    \[
    [1 - (\rho_1 + \rho_2)L + \rho_1\rho_2 L^2]z_t = (1 + \theta L)\varepsilon_t
    \]
    
    where $\varepsilon_t$ is white noise with variance $\sigma_\varepsilon^2$.
    
    \vspace{0.3em}
    
    \textbf{Identification via autocovariances}: For an MA(1), $\eta_t = (1+\theta L)\varepsilon_t$:
    \[
    \gamma_0^\eta = (1+\theta^2)\sigma_\varepsilon^2, \quad \gamma_1^\eta = \theta\sigma_\varepsilon^2
    \]
    
    From the previous exercise, the right-hand side has:
    \[
    \gamma_0^\eta = (1+\rho_2^2)\sigma_u^2 + (1+\rho_1^2)\sigma_v^2
    \]
    \[
    \gamma_1^\eta = -\rho_2\sigma_u^2 - \rho_1\sigma_v^2
    \]
\end{frame}

\begin{frame}{Medeiros ARMA: Exercise 3 --- Solution (b) Part 2}
    \textbf{Solving for $\theta$}:
    
    \vspace{0.3em}
    
    The lag-1 autocorrelation of an MA(1) is:
    \[
    \rho_1^{\text{MA}} = \frac{\gamma_1}{\gamma_0} = \frac{\theta}{1+\theta^2}
    \]
    
    Equating:
    \[
    \frac{\theta}{1+\theta^2} = \frac{-\rho_2\sigma_u^2 - \rho_1\sigma_v^2}{(1+\rho_2^2)\sigma_u^2 + (1+\rho_1^2)\sigma_v^2}
    \]
    
    \vspace{0.3em}
    
    This is a quadratic equation in $\theta$:
    \[
    \theta^2 \cdot (\text{numerador}) - \theta \cdot (\text{denominador}) + (\text{numerador}) = 0
    \]
    
    \begin{alertblock}{Two roots!}
        There are two solutions, $\theta$ and $1/\theta$, corresponding to observationally equivalent representations (invertible vs.\ non-invertible).
    \end{alertblock}
\end{frame}

\begin{frame}{Medeiros ARMA: Exercise 3 --- Solution (c)}
    \textbf{(c) Conditions for $z_t$ to be AR(2)}:
    
    \vspace{0.3em}
    
    $z_t$ is AR(2) if the MA part disappears, i.e., $\theta = 0$, which requires $\gamma_1^\eta = 0$:
    \[
    -\rho_2\sigma_u^2 - \rho_1\sigma_v^2 = 0 \quad \Rightarrow \quad \boxed{\rho_1\sigma_v^2 = -\rho_2\sigma_u^2}
    \]
    
    \vspace{0.5em}
    
    \textbf{If $\rho_1$ and $\rho_2$ have the same sign}:
    
    The left-hand side $\rho_1\sigma_v^2$ and the right-hand side $-\rho_2\sigma_u^2$ have opposite signs (since $\sigma_u^2, \sigma_v^2 > 0$).
    
    \vspace{0.3em}
    
    \begin{alertblock}{Impossibility}
        If $\text{sgn}(\rho_1) = \text{sgn}(\rho_2)$, then the condition $\rho_1\sigma_v^2 = -\rho_2\sigma_u^2$ can \textbf{never} be satisfied.
        
        \vspace{0.2em}
        
        Therefore, $z_t$ \textbf{cannot be AR(2)} when $\rho_1$ and $\rho_2$ have the same sign.
    \end{alertblock}
\end{frame}

\begin{frame}{Medeiros ARMA: Exercise 3 --- Solution (d)}
    \textbf{(d) Showing that $\theta \neq \pm 1$}:
    
    \vspace{0.5em}
    
    Suppose $\theta = 1$. Then $\rho_1^{\text{MA}} = \frac{1}{1+1} = \frac{1}{2}$.
    
    \vspace{0.3em}
    
    For this to occur, we would need:
    \[
    \frac{-\rho_2\sigma_u^2 - \rho_1\sigma_v^2}{(1+\rho_2^2)\sigma_u^2 + (1+\rho_1^2)\sigma_v^2} = \frac{1}{2}
    \]
    
    \vspace{0.3em}
    
    Since $|\rho_1|, |\rho_2| < 1$, the denominator $(1+\rho_2^2)\sigma_u^2 + (1+\rho_1^2)\sigma_v^2 > 2\sigma_u^2 + 2\sigma_v^2$ (using $\rho^2 < 1$).
    
    The numerator has absolute value $|\rho_2\sigma_u^2 + \rho_1\sigma_v^2| < \sigma_u^2 + \sigma_v^2$.
    
    \vspace{0.3em}
    
    Therefore, the absolute value of the ratio is:
    \[
    \left|\rho_1^{\text{MA}}\right| < \frac{\sigma_u^2 + \sigma_v^2}{2(\sigma_u^2 + \sigma_v^2)} = \frac{1}{2}
    \]
    
    \begin{block}{Conclusion}
        Since $|\rho_1^{\text{MA}}| < 1/2$, and $\theta = \pm 1$ would imply $|\rho_1^{\text{MA}}| = 1/2$, we have $\boxed{\theta \neq \pm 1}$.
    \end{block}
\end{frame}

\begin{frame}{Medeiros ARMA: Exercise 3 --- Solution (e)}
    \textbf{(e) Forecasting using the ARMA(2,1)}:
    
    \vspace{0.3em}
    
    Do modelo $[1 - (\rho_1+\rho_2)L + \rho_1\rho_2 L^2]z_t = (1+\theta L)\varepsilon_t$:
    \[
    z_t = (\rho_1+\rho_2)z_{t-1} - \rho_1\rho_2 z_{t-2} + \varepsilon_t + \theta\varepsilon_{t-1}
    \]
    
    \textbf{Forecast}:
    \[
    \E(z_{T+1}|\mathcal{F}_T) = (\rho_1+\rho_2)z_T - \rho_1\rho_2 z_{T-1} + \theta\varepsilon_T
    \]
    
    \textbf{Error}: $e_{T+1} = z_{T+1} - \E(z_{T+1}|\mathcal{F}_T) = \varepsilon_{T+1}$
    
    \textbf{Error variance} $V_A$:
    \[
    V_A = \Var(\varepsilon_{T+1}) = \sigma_\varepsilon^2
    \]
    
    \vspace{0.3em}
    
    \textbf{Comparing $V_A - V_D$}: $V_A = \sigma_\varepsilon^2$ and $V_D = \sigma_u^2 + \sigma_v^2$.
    
    Since $(1+\theta^2)\sigma_\varepsilon^2 = (1+\rho_2^2)\sigma_u^2 + (1+\rho_1^2)\sigma_v^2$, and $\theta^2 < 1$, $\rho_i^2 < 1$, we have $V_A \approx V_D$ (depending on the specific parameter values).
\end{frame}

% ----------------------------------------------------------------------------
\begin{frame}{Medeiros ARMA: Exercise 6 --- Statement}
    \begin{block}{Partial Adjustment Model}
        The desired level $y_t^*$ is related to $x_t$ by: $y_t^* = \beta x_t$.
        
        $x_t$ is covariance-stationary. A fraction $\gamma$ of the disequilibrium is removed through partial adjustment:
        \[
        y_t - y_{t-1} = \gamma(y_t^* - y_{t-1}) + e_t, \quad e_t \sim N(0, \sigma_e^2), \; 0 < \gamma < 1
        \]
    \end{block}
    
    \textbf{Compare with the adaptive expectations model}:
    \[
    y_t = \beta x_{t+1}^e + v_t, \quad v_t \sim N(0, \sigma_v^2)
    \]
    \[
    x_{t+1}^e - x_t^e = \lambda(x_t - x_t^e), \quad 0 < \lambda < 1
    \]
    
    \vspace{0.3em}
    
    \textbf{Tasks}: (a) Compare the models. (b) Show that OLS of $y_t$ on $y_{t-1}$ and $x_t$ is consistent for the model (1)-(2), but inconsistent for (3)-(4).
\end{frame}

\begin{frame}{Medeiros ARMA: Exercise 6 --- Solution (a)}
    \textbf{(a) Comparing the two models}:
    
    \vspace{0.3em}
    
    \textbf{Partial Adjustment Model}: Substituting $y_t^* = \beta x_t$:
    \begin{align*}
    y_t - y_{t-1} &= \gamma(\beta x_t - y_{t-1}) + e_t \\
    y_t &= \gamma\beta x_t + (1-\gamma)y_{t-1} + e_t
    \end{align*}
    
    \textbf{Adaptive Expectations Model}: Solving for $x_{t+1}^e$:
    \[
    x_{t+1}^e = \lambda x_t + (1-\lambda)x_t^e = \lambda \sum_{j=0}^{\infty}(1-\lambda)^j x_{t-j}
    \]
    
    Substituting into $y_t = \beta x_{t+1}^e + v_t$:
    \[
    y_t = \beta\lambda \sum_{j=0}^{\infty}(1-\lambda)^j x_{t-j} + v_t
    \]
    
    Using the Koyck transformation: $y_t = \beta\lambda x_t + (1-\lambda)y_{t-1} + v_t - (1-\lambda)v_{t-1}$
\end{frame}

\begin{frame}{Medeiros ARMA: Exercise 6 --- Solution (a) Cont.}
    \textbf{Reduced-form representation of both models}:
    
    \vspace{0.5em}
    
    \begin{tabular}{lcc}
    \toprule
    & \textbf{Partial Adjustment} & \textbf{Adaptive Expectations} \\
    \midrule
    Form & $y_t = \alpha x_t + \phi y_{t-1} + e_t$ & $y_t = \alpha x_t + \phi y_{t-1} + \eta_t$ \\
    $x_t$ coeff. & $\alpha = \gamma\beta$ & $\alpha = \beta\lambda$ \\
    $y_{t-1}$ coeff. & $\phi = 1-\gamma$ & $\phi = 1-\lambda$ \\
    Error & $e_t$ (WN) & $\eta_t = v_t - (1-\lambda)v_{t-1}$ (MA(1)!) \\
    \bottomrule
    \end{tabular}
    
    \vspace{0.5em}
    
    \begin{alertblock}{Key Difference}
        \begin{itemize}
            \item In partial adjustment: $e_t$ is white noise $\Rightarrow$ exogenous
            \item In adaptive expectations: $\eta_t$ is MA(1) $\Rightarrow$ correlated with $y_{t-1}$!
        \end{itemize}
    \end{alertblock}
\end{frame}

\begin{frame}{Medeiros ARMA: Exercise 6 --- Solution (b)}
    \textbf{(b) OLS consistency}:
    
    \vspace{0.3em}
    
    Consider the regression: $y_t = \alpha x_t + \phi y_{t-1} + u_t$
    
    \vspace{0.3em}
    
    \textbf{Partial Adjustment Model}: $u_t = e_t$ is white noise.
    \begin{itemize}
        \item $\Cov(y_{t-1}, e_t) = 0$ (error is uncorrelated with past variables)
        \item $\Cov(x_t, e_t) = 0$ (exogeneidade estrita)
        \item $\Rightarrow$ OLS is \textbf{consistent} $\checkmark$
    \end{itemize}
    
    \vspace{0.5em}
    
    \textbf{Adaptive Expectations Model}: $u_t = \eta_t = v_t - (1-\lambda)v_{t-1}$
    \begin{itemize}
        \item $y_{t-1}$ depends on $v_{t-1}$ (through the recursive equation)
        \item $\Cov(y_{t-1}, \eta_t) = \Cov(y_{t-1}, -(1-\lambda)v_{t-1}) \neq 0$
        \item $\Rightarrow$ OLS is \textbf{inconsistent} $\times$
    \end{itemize}
    
    \begin{block}{Intuition}
        The regressor $y_{t-1}$ is correlated with the error $\eta_t$ due to the MA(1) component.
    \end{block}
\end{frame}

\begin{frame}{Medeiros ARMA: Exercise 6 --- Solution (c)}
    \textbf{(c) How to check which model is appropriate?}
    
    \vspace{0.5em}
    
    \textbf{Strategy}: Test for autocorrelation in the residuals.
    
    \vspace{0.3em}
    
    \begin{enumerate}
        \item \textbf{Estimate by OLS}: $y_t = \hat{\alpha} x_t + \hat{\phi} y_{t-1} + \hat{u}_t$
        
        \vspace{0.3em}
        
        \item \textbf{Examine the residuals} $\hat{u}_t$:
        \begin{itemize}
            \item If $\hat{u}_t$ is white noise $\Rightarrow$ Partial Adjustment Model
            \item If $\hat{u}_t$ shows autocorrelation (MA(1)) $\Rightarrow$ Adaptive Expectations
        \end{itemize}
        
        \vspace{0.3em}
        
        \item \textbf{Formal tests}:
        \begin{itemize}
            \item Durbin--Watson test (for AR(1) in the residuals)
            \item Breusch--Godfrey test (for higher-order autocorrelation)
            \item Ljung--Box test on the residual ACF
        \end{itemize}
    \end{enumerate}
    
    \vspace{0.3em}
    
    \begin{alertblock}{Caution}
        If the correct model is adaptive expectations, use appropriate estimation methods (GMM, NLLS, or MLE) instead of OLS.
    \end{alertblock}
\end{frame}

% ----------------------------------------------------------------------------
% MEDEIROS - VECTOR PROCESSES (Exercise 4)
% ----------------------------------------------------------------------------
\begin{frame}{Medeiros Vector Processes: Exercise 4 --- Statement}
    \begin{block}{Macroeconomic Model}
        \begin{align*}
        (1): & \quad \pi_t = \lambda y_t + \pi_t^e + u_{1t}, \quad 0 < \lambda < 1 \\
        (2): & \quad y_t = \gamma(i_{t-1} - \pi_t^e) + u_{2,t}, \quad -1 < \gamma < 0 \\
        (3): & \quad \pi_t^e = \pi_{t-1} \\
        (4): & \quad i_t = i^* + \rho(\pi_t - \pi^*), \quad \rho \geq 0
        \end{align*}
        where $\pi_t$ = inflation, $y_t$ = output gap, $i_t$ = nominal interest rate, $(u_{1t}, u_{2t})' \sim N(0, \Sigma)$.
    \end{block}
    
    \vspace{0.3em}
    
    \textbf{Interpretation}: (1) Phillips curve, (2) IS, (3) adaptive expectations, (4) Taylor rule.
\end{frame}

\begin{frame}{Medeiros Vector Processes: Exercise 4 --- Statement (cont.)}
    \textbf{Tasks}:
    \begin{enumerate}[(a)]
        \item Show that $\pi_t$ follows an ARMA$(p,q)$. Relate the coefficients to the structural parameters.
        
        \item Consider $\rho = 0, 1, 2$. What can be said about inflation and monetary policy? Simulate 200 observations in MATLAB.
        
        \item Compute $\hat{\pi}_{t+j|t} = \E[\pi_{t+j}|\mathcal{F}_t]$ and $\Var(\pi_{t+j} - \hat{\pi}_{t+j|t})$.
        
        \item Let $\mathbf{x}_t = (\pi_t, y_t)'$. Find $\mathbb{D}(\mathbf{x}_t|\mathbf{x}_{t-1})$ and $\E(\pi_t|y_t, \mathbf{x}_{t-1})$.
        
        \item What would change if $\pi_t^e = \alpha\pi_{t-1} + (1-\alpha)\pi_{t-1}^e$?
    \end{enumerate}
\end{frame}

\begin{frame}{Medeiros Vector Processes: Exercise 4 --- Solution (a) Part 1}
    \textbf{(a) Deriving the ARMA process for $\pi_t$}:
    
    \vspace{0.3em}
    
    \textbf{Step 1}: Substitute (3) and (4) into (2):
    \begin{align*}
    y_t &= \gamma(i_{t-1} - \pi_{t-1}) + u_{2t} \\
    &= \gamma(i^* + \rho(\pi_{t-1} - \pi^*) - \pi_{t-1}) + u_{2t} \\
    &= \gamma(i^* - \rho\pi^*) + \gamma(\rho - 1)\pi_{t-1} + u_{2t}
    \end{align*}
    
    \vspace{0.3em}
    
    \textbf{Step 2}: Substitute into (1):
    \begin{align*}
    \pi_t &= \lambda[\gamma(i^* - \rho\pi^*) + \gamma(\rho-1)\pi_{t-1} + u_{2t}] + \pi_{t-1} + u_{1t} \\
    &= c + [1 + \lambda\gamma(\rho-1)]\pi_{t-1} + u_{1t} + \lambda u_{2t}
    \end{align*}
    
    where $c = \lambda\gamma(i^* - \rho\pi^*)$ is a constant.
\end{frame}

\begin{frame}{Medeiros Vector Processes: Exercise 4 --- Solution (a) Part 2}
    \textbf{Result}: $\pi_t$ follows an \textbf{AR(1)} process:
    \[
    \boxed{\pi_t = c + \phi \pi_{t-1} + \varepsilon_t}
    \]
    
    where:
    \begin{itemize}
        \item $\phi = 1 + \lambda\gamma(\rho - 1)$
        \item $\varepsilon_t = u_{1t} + \lambda u_{2t}$ (a linear combination of the shocks)
        \item $\Var(\varepsilon_t) = \sigma_{11}^2 + 2\lambda\sigma_{12} + \lambda^2\sigma_{22}^2$
    \end{itemize}
    
    \vspace{0.5em}
    
    \begin{block}{Link to Structural Parameters}
        \begin{itemize}
            \item $\phi < 1$ (stationarity) $\Leftrightarrow$ $\lambda\gamma(\rho-1) < 0$
            \item Since $\lambda > 0$ and $\gamma < 0$: $\phi < 1$ $\Leftrightarrow$ $\rho > 1$
            \item This is the \textbf{Taylor principle}: $\rho > 1$ stabilizes inflation!
        \end{itemize}
    \end{block}
\end{frame}

\begin{frame}{Medeiros Vector Processes: Exercise 4 --- Solution (b)}
    \textbf{(b) Analysis for different values of $\rho$}:
    
    \vspace{0.3em}
    
    Suppose $\lambda = 0.5$, $\gamma = -0.5$. Then $\phi = 1 + 0.5 \cdot (-0.5) \cdot (\rho - 1) = 1 - 0.25(\rho - 1)$.
    
    \vspace{0.5em}
    
    \begin{tabular}{cccc}
    \toprule
    $\rho$ & $\phi$ & Stationary? & Interpretation \\
    \midrule
    0 & 1.25 & No & Explosive (violates stability/unit-root condition) \\
    1 & 1.00 & No & Random walk \\
    2 & 0.75 & Yes & Stationary inflation \\
    \bottomrule
    \end{tabular}
    
    \vspace{0.5em}
    
    \begin{alertblock}{Economic Interpretation}
        \begin{itemize}
            \item $\rho = 0$: the central bank does not respond to inflation $\Rightarrow$ explosive inflation
            \item $\rho = 1$: the central bank responds one-for-one $\Rightarrow$ constant real rate, inflation is I(1)
            \item $\rho > 1$: the central bank responds more than one-for-one $\Rightarrow$ stationary inflation
        \end{itemize}
    \end{alertblock}
\end{frame}

\begin{frame}{Medeiros Vector Processes: Exercise 4 --- Solution (c)}
    \textbf{(c) Forecasts and forecast error variance}:
    
    \vspace{0.3em}
    
    For the AR(1) $\pi_t = c + \phi\pi_{t-1} + \varepsilon_t$ (assuming $|\phi| < 1$):
    
    \vspace{0.3em}
    
    \textbf{$j$-step-ahead forecast}:
    \[
    \hat{\pi}_{t+j|t} = \mu(1 - \phi^j) + \phi^j \pi_t
    \]
    where $\mu = c/(1-\phi)$ is the unconditional mean.
    
    \vspace{0.3em}
    
    \textbf{Forecast error variance}:
    \[
    \Var(\pi_{t+j} - \hat{\pi}_{t+j|t}) = \sigma_\varepsilon^2 \sum_{i=0}^{j-1}\phi^{2i} = \sigma_\varepsilon^2 \cdot \frac{1 - \phi^{2j}}{1 - \phi^2}
    \]
    
    \vspace{0.3em}
    
    \textbf{Limit}: As $j \to \infty$:
    \[
    \lim_{j\to\infty} \Var(\pi_{t+j} - \hat{\pi}_{t+j|t}) = \frac{\sigma_\varepsilon^2}{1 - \phi^2} = \Var(\pi_t)
    \]
\end{frame}

\begin{frame}{Medeiros Vector Processes: Exercise 4 --- Solution (d)}
    \textbf{(d) Conditional distribution of $\mathbf{x}_t = (\pi_t, y_t)'$}:
    
    \vspace{0.3em}
    
    From the model equations:
    \begin{align*}
    \pi_t &= c_\pi + \phi \pi_{t-1} + \varepsilon_t \\
    y_t &= c_y + \gamma(\rho-1)\pi_{t-1} + u_{2t}
    \end{align*}
    
    Therefore, $\mathbf{x}_t | \mathbf{x}_{t-1} \sim N(\boldsymbol{\mu}_t, \boldsymbol{\Sigma})$, where:
    \[
    \boldsymbol{\mu}_t = \begin{pmatrix} c_\pi + \phi\pi_{t-1} \\ c_y + \gamma(\rho-1)\pi_{t-1} \end{pmatrix}, \quad
    \boldsymbol{\Sigma} = \begin{pmatrix} \sigma_\varepsilon^2 & \lambda\sigma_{22}^2 + \sigma_{12} \\ \lambda\sigma_{22}^2 + \sigma_{12} & \sigma_{22}^2 \end{pmatrix}
    \]
    
    \vspace{0.3em}
    
    \textbf{Conditional expectation} $\E(\pi_t | y_t, \mathbf{x}_{t-1})$:
    
    Using the conditional normal formula:
    \[
    \E(\pi_t | y_t, \mathbf{x}_{t-1}) = \E(\pi_t|\mathbf{x}_{t-1}) + \frac{\Cov(\pi_t, y_t)}{\Var(y_t)}[y_t - \E(y_t|\mathbf{x}_{t-1})]
    \]
\end{frame}

\begin{frame}{Medeiros Vector Processes: Exercise 4 --- Solution (e)}
    \textbf{(e) Generalized adaptive expectations}: $\pi_t^e = \alpha\pi_{t-1} + (1-\alpha)\pi_{t-1}^e$
    
    \vspace{0.3em}
    
    Solving recursively:
    \[
    \pi_t^e = \alpha \sum_{j=0}^{\infty}(1-\alpha)^j \pi_{t-1-j}
    \]
    
    \vspace{0.3em}
    
    \textbf{Effect on the model}: inflation now depends on the entire past history, not only on $\pi_{t-1}$.
    
    \vspace{0.3em}
    
    Using the Koyck transformation:
    \[
    \pi_t^e = \alpha\pi_{t-1} + (1-\alpha)\pi_{t-1}^e
    \]
    
    \vspace{0.3em}
    
    The process for $\pi_t$ becomes an \textbf{ARMA(2,1)} instead of AR(1):
    \[
    \pi_t = c + \phi_1\pi_{t-1} + \phi_2\pi_{t-2} + \varepsilon_t + \theta\varepsilon_{t-1}
    \]
    
    where the coefficients depend on $\alpha, \lambda, \gamma, \rho$ in a more complex way.
\end{frame}

% ----------------------------------------------------------------------------
\begin{frame}{Exercise Session: Summary}
    \textbf{Key Skills Practiced}:
    
    \vspace{0.5em}
    
    \begin{enumerate}
        \item Checking stationarity conditions (roots outside unit circle)
        \item Computing autocovariances and ACF for ARMA processes
        \item Converting between AR and MA representations
        \item Identifying common factors and parameter redundancy
        \item Working with vector processes
    \end{enumerate}
    
    \vspace{1em}
    
    \begin{alertblock}{For Next Session}
        Review these exercises and attempt additional problems from the appendix for extra practice.
    \end{alertblock}
\end{frame}

% ============================================================================
% APPENDIX
% ============================================================================
\appendix

\begin{frame}[plain]
    \begin{center}
        {\Large \textbf{Appendix}}\\[1em]
        {\large Detailed Material and Extended Examples}
    \end{center}
\end{frame}

% ============================================================================
% A1: LOCAL IDENTIFICATION - INFORMATION MATRIX
% ============================================================================
\begin{frame}{Checking Local Identification: The Information Matrix}
    \label{slide:info_matrix}
    \textbf{Idea}: Use the Kullback-Leibler divergence between distributions:
    \[
    \Delta_{KL}(\theta \mid \theta_0) = -\int \log\left(\frac{\mathbb{D}(\mathbf{Y}^T \mid \theta)}{\mathbb{D}(\mathbf{Y}^T \mid \theta_0)}\right) \mathbb{D}(\mathbf{Y}^T \mid \theta_0) \, d\mathbf{Y}^T
    \]
    
    \vspace{0.3em}
    
    By Jensen's inequality: $\Delta_{KL}(\theta \mid \theta_0) \geq 0$, with equality iff distributions are equal.
    
    \vspace{0.5em}
    
    \textbf{Second-order condition}: The Hessian at $\theta_0$ is the \textbf{Fisher Information Matrix}:
    \[
    \mathcal{I}(\theta_0) = -\E_{\theta_0}\left[\nabla_{\theta^2} \log \mathbb{D}(\mathbf{Y}^T \mid \theta_0)\right]
    \]
    
    \vspace{0.3em}
    
    \begin{theorem}[Local Identification Condition]
        If $\mathcal{I}(\theta_0)$ is \textbf{positive definite} (full rank), then $\theta_0$ is locally identified.
    \end{theorem}
    
    \vspace{0.3em}
    
    \hfill \hyperlink{slide:global_local_id}{\beamerbutton{$\leftarrow$ Back to main slides}}
\end{frame}

% ============================================================================
% A2: AR(1) FIGURES
% ============================================================================
\begin{frame}{AR(1) Inflation Example: Figures}
    \label{slide:ar1_figures}
    
    \textbf{Figure 1}: Sensitivity of $\phi_1$ to Taylor Rule coefficient $\rho$
    
    \vspace{0.5em}
    
    % PLACEHOLDER: Include figure from metrics bible
    % \includegraphics[width=0.8\textwidth]{Figures/param_combination.png}
    
    \begin{center}
        \textit{[Figure: param\_combination.png from metrics bible]}
    \end{center}
    
    \vspace{0.5em}
    
    Under the simple monetary policy model with $\lambda = 0.5$, $\gamma = -0.5$:
    \begin{itemize}
        \item Stability requires $1 < \rho < 9$
        \item Taylor principle ($\rho > 1$) ensures stationary inflation
    \end{itemize}
    
    \hfill \hyperlink{slide:global_local_id}{\beamerbutton{$\leftarrow$ Back}}
\end{frame}

% ----------------------------------------------------------------------------
\begin{frame}{AR(1) Inflation: ACF Patterns}
    \textbf{Figure 2}: ACF under different Taylor Rule calibrations
    
    \vspace{0.5em}
    
    % PLACEHOLDER: Include figure from metrics bible
    % \includegraphics[width=0.45\textwidth]{Figures/acf_pi_1.png}
    % \includegraphics[width=0.45\textwidth]{Figures/acf_pi_2.png}
    
    \begin{center}
        \textit{[Figure: acf\_pi\_1.png, acf\_pi\_2.png from metrics bible]}
    \end{center}
    
    \vspace{0.5em}
    
    \textbf{Key observations}:
    \begin{itemize}
        \item $\rho \in \{2,3,4\}$: Persistent, non-oscillatory decay
        \item $\rho \in \{6,7,8\}$: Faster decay, potentially oscillating
    \end{itemize}
\end{frame}

% ============================================================================
% A3: DETAILED PROOFS
% ============================================================================
\begin{frame}[plain]
    \begin{center}
        {\Large \textbf{A3: Detailed Proofs}}
    \end{center}
\end{frame}

% ----------------------------------------------------------------------------
\begin{frame}{Proof: MA($q$) Stationarity}
    \textbf{Claim}: All finite-order MA($q$) processes are covariance-stationary.
    
    \vspace{0.5em}
    
    \textbf{Proof}:
    
    \textbf{Mean}:
    \[
    \E[y_t] = \E\left[u_t + \sum_{j=1}^q \theta_j u_{t-j} + \theta_0\right] = \theta_0
    \]
    
    \textbf{Autocovariance}:
    \begin{align*}
        \Gamma_k &= \E[(y_t - \mu)(y_{t-k} - \mu)] \\
        &= \E\left[\left(\sum_{j=0}^q \theta_j u_{t-j}\right)\left(\sum_{m=0}^q \theta_m u_{t-k-m}\right)\right]
    \end{align*}
    
    Since $\E[u_s u_r] = \sigma^2 \mathbf{1}_{s=r}$:
    \[
    \Gamma_k = \begin{cases} \left(\theta_k + \sum_{j=1}^{q-k}\theta_j\theta_{j+k}\right)\sigma^2 & k \leq q \\ 0 & k > q \end{cases}
    \]
    
    Both moments are \textbf{independent of $t$}. $\square$
\end{frame}

% ----------------------------------------------------------------------------
\begin{frame}{Proof: Invertibility Condition for MA($q$)}
    \textbf{Claim}: MA($q$) is invertible iff roots of $\theta_q(z) = 0$ lie outside unit circle.
    
    \vspace{0.3em}
    
    \textbf{Proof}: Start with $y_t = \theta_0 + \theta_q(L)u_t$. We need:
    \[
    u_t = \theta_q(L)^{-1}(y_t - \theta_0)
    \]
    
    Factor the polynomial: $\theta_q(L) = (1-\lambda_1 L)(1-\lambda_2 L)\cdots(1-\lambda_q L)$
    
    where $\lambda_i$ are reciprocals of the roots.
    
    \vspace{0.3em}
    
    For each factor: $(1-\lambda_i L)^{-1} = \sum_{k=0}^{\infty}(\lambda_i L)^k$
    
    This converges iff $|\lambda_i| < 1$, i.e., roots $|1/\lambda_i| > 1$ (outside UC).
    
    \vspace{0.3em}
    
    If convergent:
    \[
    \theta_q(L)^{-1} = \sum_{j=0}^{\infty}\psi_j L^j \quad \Rightarrow \quad u_t = \sum_{j=0}^{\infty}\psi_j(y_{t-j} - \theta_0) \in \Span\{y_t, y_{t-1}, \ldots\}
    \]
    $\square$
\end{frame}

% ----------------------------------------------------------------------------
\begin{frame}{MA($\infty$) Representation of AR($p$)}
    \textbf{Claim}: A stationary AR($p$) can be written as MA($\infty$).
    
    \vspace{0.5em}
    
    \textbf{Proof} (AR(1) case): Let $y_t = \phi_1 y_{t-1} + u_t$ with $|\phi_1| < 1$.
    
    Iterating backward (assuming process started infinitely far in the past):
    \begin{align*}
        y_t &= \phi_1 y_{t-1} + u_t \\
        &= \phi_1(\phi_1 y_{t-2} + u_{t-1}) + u_t \\
        &= \cdots \\
        &= \sum_{i=0}^{\infty} \phi_1^i u_{t-i} + \lim_{n\to\infty} \phi_1^n y_{t-n}
    \end{align*}
    
    Since $|\phi_1| < 1$: $\phi_1^n \to 0$ and $\sum|\phi_1|^i < \infty$.
    
    \vspace{0.3em}
    
    Therefore: $y_t = \sum_{i=0}^{\infty}\psi_i u_{t-i}$ with $\psi_i = \phi_1^i$.
    
    \vspace{0.3em}
    
    \textbf{General case}: Use companion form and iterate. $\square$
\end{frame}

% ============================================================================
% A4: EXTENDED EXAMPLES
% ============================================================================
\begin{frame}[plain]
    \begin{center}
        {\Large \textbf{A4: Extended Examples}}
    \end{center}
\end{frame}

% ----------------------------------------------------------------------------
\begin{frame}{Example: Observational Equivalence in MA(2)}
    Consider MA(2): $y_t = u_t + \theta_1 u_{t-1} + \theta_2 u_{t-2}$
    
    \vspace{0.3em}
    
    Factor: $\theta_2(z) = 1 + \theta_1 z + \theta_2 z^2 = (1-\lambda_1 z)(1-\lambda_2 z)$
    
    where $\theta_1 = -(\lambda_1 + \lambda_2)$, $\theta_2 = \lambda_1\lambda_2$
    
    \vspace{0.5em}
    
    \textbf{Invertible representation}: Roots outside UC $\Rightarrow$ $|\lambda_i| < 1$
    
    \textbf{Non-invertible alternative}: ``Flip'' each root:
    \[
    \lambda_i^* = 1/\lambda_i \quad \Rightarrow \quad \theta_1^* = \theta_1/\theta_2, \quad \theta_2^* = 1/\theta_2
    \]
    
    \textbf{Match ACGF}: Scale variance by $\sigma^{*2} = \sigma^2 \theta_2^2$
    
    \vspace{0.5em}
    
    \textbf{Result}: These have \textbf{identical} autocovariance structure but different impulse responses!
\end{frame}

% ----------------------------------------------------------------------------
\begin{frame}{Example: NK Model Invertibility (Wolf, 2022)}
    \textbf{Three-equation NK model} with demand, supply, and monetary shocks:
    
    \[
    \begin{pmatrix} y_t \\ \pi_t \\ i_t \end{pmatrix} = \frac{1}{1+\phi_\pi\kappa} \begin{pmatrix} \sigma^d & \phi_\pi\sigma^s & -\sigma^m \\ \kappa\sigma^d & -\sigma^s & -\kappa\sigma^m \\ \phi_\pi\kappa\sigma^d & -\phi_\pi\sigma^s & \sigma^m \end{pmatrix} \begin{pmatrix} \varepsilon_t^d \\ \varepsilon_t^s \\ \varepsilon_t^m \end{pmatrix}
    \]
    
    \vspace{0.3em}
    
    \textbf{With all 3 observables}: Impact matrix $\Theta$ is invertible for standard parameters.
    
    \vspace{0.3em}
    
    \textbf{With only $y_t$ and $\pi_t$}:
    \begin{itemize}
        \item Can back out $\varepsilon_t^s$ (supply shock)
        \item \textbf{Cannot} disentangle $\varepsilon_t^d$ and $\varepsilon_t^m$!
    \end{itemize}
    
    \vspace{0.3em}
    
    \begin{alertblock}{Lesson}
        Even in well-specified models, fewer observables than shocks $\Rightarrow$ non-invertibility!
    \end{alertblock}
\end{frame}

% ============================================================================
% A5: REFERENCE MATERIAL
% ============================================================================
\begin{frame}[plain]
    \begin{center}
        {\Large \textbf{A5: Reference Material}}
    \end{center}
\end{frame}

% ----------------------------------------------------------------------------
\begin{frame}{Key Formulas: Quick Reference}
    \textbf{AR($p$)}:
    \begin{itemize}
        \item Process: $\Phi_p(L)y_t = \phi_0 + u_t$
        \item Stationarity: Roots of $\Phi_p(z) = 0$ outside UC
        \item Mean: $\mu = \phi_0/(1-\sum\phi_i)$
        \item Yule-Walker: $\gamma_k = \sum_{i=1}^p \phi_i \gamma_{k-i}$
    \end{itemize}
    
    \vspace{0.5em}
    
    \textbf{MA($q$)}:
    \begin{itemize}
        \item Process: $y_t = \theta_0 + \Theta_q(L)u_t$
        \item Always stationary
        \item Invertibility: Roots of $\Theta_q(z) = 0$ outside UC
        \item $\gamma_k = 0$ for $k > q$ (cutoff)
    \end{itemize}
    
    \vspace{0.5em}
    
    \textbf{ARMA($p,q$)}:
    \begin{itemize}
        \item Process: $\Phi_p(L)y_t = \phi_0 + \Theta_q(L)u_t$
        \item ACGF: $\Gamma(z) = \frac{\Theta_q(z)\Theta_q(z^{-1})}{\Phi_p(z)\Phi_p(z^{-1})}\sigma^2$
    \end{itemize}
\end{frame}

% ----------------------------------------------------------------------------
\begin{frame}{References}
    \small
    \begin{itemize}
        \item Fernández-Villaverde, Rubio-Ramírez, Schorfheide (2016). ``Solution and Estimation Methods for DSGE Models.'' \textit{Handbook of Macroeconomics}
        
        \item Hamilton (1994). \textit{Time Series Analysis}. Princeton University Press
        
        \item Wolf (2022). ``Linear Time Series'' (MIT 14.461 Lecture Notes)
        
        \item Komunjer, Ng (2011). ``Dynamic Identification of DSGE Models.'' \textit{Econometrica}
        
        \item Lecture Notes in Econometrics (Lincoln, 2025). Chapter 7: Time Series
        
        \item Lecture Notes in Macroeconomics (Lincoln, 2025). Chapter 21: Identification
    \end{itemize}
\end{frame}

\end{document}
