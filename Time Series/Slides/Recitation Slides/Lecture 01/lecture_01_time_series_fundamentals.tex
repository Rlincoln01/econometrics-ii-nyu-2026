\documentclass[aspectratio=169,11pt]{beamer}

% Theme and color setup
\usetheme{default}
\usecolortheme{default}

% Remove navigation symbols
\setbeamertemplate{navigation symbols}{}

% Custom colors (MIT-inspired)
\definecolor{mitred}{RGB}{163,31,52}
\definecolor{darkblue}{RGB}{0,40,85}
\definecolor{lightgray}{RGB}{245,245,245}
\definecolor{darkgray}{RGB}{64,64,64}

% Set main colors
\setbeamercolor{frametitle}{fg=white,bg=darkblue}
\setbeamercolor{title}{fg=darkblue}
\setbeamercolor{author}{fg=black}
\setbeamercolor{institute}{fg=black}
\setbeamercolor{date}{fg=black}
\setbeamercolor{structure}{fg=darkblue}
\setbeamercolor{normal text}{fg=black}
\setbeamercolor{itemize item}{fg=mitred}
\setbeamercolor{itemize subitem}{fg=darkblue}
\setbeamercolor{enumerate item}{fg=darkblue}

% Frame title band
\newlength{\titlebandht}
\setlength{\titlebandht}{4ex}

\setbeamertemplate{frametitle}{
  \nointerlineskip
  \begin{beamercolorbox}[wd=\paperwidth,ht=\titlebandht,dp=1.1ex,leftskip=1.2em,rightskip=1.2em]{frametitle}
    \usebeamerfont{frametitle}\insertframetitle
    \ifx\insertframesubtitle\@empty\relax\else\\[-0.2ex]
      \usebeamerfont{framesubtitle}\insertframesubtitle
    \fi
  \end{beamercolorbox}
  \vspace{0.6em}
}

% Font settings
\usefonttheme{professionalfonts}
\usefonttheme{serif}

% Footline with page numbers
\setbeamertemplate{footline}{
    \hfill\insertframenumber\hspace{2em}\vspace{0.5em}
}

% Packages
\usepackage{amsmath,amssymb,amsthm}
\usepackage{graphicx}
\usepackage{tikz}
\usepackage{booktabs}
\usepackage{mathtools}
\usepackage{bm}
\usepackage{bbm}

% Custom commands for math
\newcommand{\E}{\mathbb{E}}
\renewcommand{\P}{\mathbb{P}}
\newcommand{\R}{\mathbb{R}}
\newcommand{\N}{\mathbb{N}}
\newcommand{\Z}{\mathbb{Z}}
\newcommand{\F}{\mathcal{F}}
\newcommand{\Cov}{\text{Cov}}
\newcommand{\Var}{\text{Var}}
\newcommand{\Corr}{\text{Corr}}
\newcommand{\xbf}{\mathbf{x}}
\newcommand{\ybf}{\mathbf{y}}
\newcommand{\zbf}{\mathbf{z}}
\newcommand{\ubf}{\mathbf{u}}
\newcommand{\vbf}{\mathbf{v}}
\DeclareMathOperator*{\argmin}{arg\,min}
\DeclareMathOperator*{\argmax}{arg\,max}
\DeclareMathOperator*{\plim}{plim}

% Theorem environments
\setbeamertemplate{theorems}[numbered]
\theoremstyle{plain}
\newtheorem{proposition}{Proposition}
\newtheorem{assumption}{Assumption}
\theoremstyle{definition}

% Block colors
\setbeamercolor{block title}{bg=lightgray,fg=darkblue}
\setbeamercolor{block body}{bg=lightgray!50}

% Alert block colors
\setbeamercolor{block title alerted}{bg=mitred!15,fg=mitred}
\setbeamercolor{block body alerted}{bg=mitred!5}

% Title information
\title{Time Series Fundamentals}
\subtitle{TA Session 1: Introduction to Stochastic Processes}
\author{Rafael Lincoln}
\institute{Econometrics II -- NYU PhD Program}
\date{Spring 2026}

\begin{document}

% ============================================================================
% TITLE SLIDE
% ============================================================================
\begin{frame}[plain]
    \titlepage
\end{frame}

% ============================================================================
% OUTLINE
% ============================================================================
\begin{frame}{Today's Roadmap}
    \begin{enumerate}
        \item \textbf{Introduction: Why Time Series?}
        \begin{itemize}
            \item Objectives: Dynamic relationships \& Forecasting
            \item Structural vs. Reduced-form models
            \item Stylized facts
        \end{itemize}
        \vspace{0.5em}
        \item \textbf{Introduction to Stochastic Processes}
        \begin{itemize}
            \item Formal definitions: Filtrations, measurability
            \item Models: DGP, correct specification
            \item Key restrictions: Stationarity
            \item Linear operators (lag \& difference)
        \end{itemize}
        \vspace{0.5em}
        \item \textbf{Examples from Macroeconomics}
        \begin{itemize}
            \item Monetary policy transmission
            \item Samuelson's multiplier-accelerator
            \item Euler equations
        \end{itemize}
    \end{enumerate}
\end{frame}

% ============================================================================
% PART I: INTRODUCTION
% ============================================================================
\begin{frame}[plain]
    \begin{center}
        {\Large \textbf{Part I: Introduction}}\\[1em]
        {\large Why Time Series?}
    \end{center}
\end{frame}

% ----------------------------------------------------------------------------
\begin{frame}{What is a Time Series?}
    \begin{definition}
        A \textbf{time series} is a sequence of data points indexed in time order, taken at successive equally spaced points in time.
    \end{definition}
    
    \vspace{1em}
    
    \textbf{How does it differ from other data structures?}
    \begin{itemize}
        \item \textbf{Cross-sectional data}: No natural ordering of observations
        \item \textbf{Spatial data}: Ordering based on geographic location
        \item \textbf{Time series}: Natural temporal ordering with dependence structure
    \end{itemize}
    
    \vspace{1em}
    
    \begin{alertblock}{Key Challenge}
        We observe only \textit{one realization} of history. Inference requires assumptions ensuring ``the present is like the past.''
    \end{alertblock}
\end{frame}

% ----------------------------------------------------------------------------
\begin{frame}{Why Should We Care About Time Series?}
    \textbf{Two Main Objectives:}
    
    \vspace{0.5em}
    
    \begin{enumerate}
        \item \textbf{Dynamic Relationships} (Correlation and/or Causality)
        \begin{itemize}
            \item \textit{Causality}: Impact evaluation, policy analysis, price optimization
            \item \textit{Correlation}: Comovement between variables (balanced growth, asset pricing)
        \end{itemize}
        
        \vspace{1em}
        
        \item \textbf{Forecasting} future values based on scenarios
        \begin{itemize}
            \item Macroeconomic forecasting (GDP, inflation, unemployment)
            \item Asset allocation and risk quantification
            \item \textit{Nowcasting}: High-frequency monitoring of low-frequency variables
        \end{itemize}
    \end{enumerate}
\end{frame}

% ----------------------------------------------------------------------------
\begin{frame}{Example: Generate HD with structural model}
    \vspace{-0.2em}
    \begin{block}{Smets \& Wouters (2007)}
        Structural model $\Rightarrow$ identified shocks $\Rightarrow$ historical decomposition of inflation and output growth over time.
    \end{block}
    \begin{columns}[T]
        \begin{column}{0.49\textwidth}
            \centering
            % TODO: provide actual figure file (e.g., figures/sw_inflation.png)
            \includegraphics[width=\textwidth]{figures/SM_CPI_HD.png}
        \end{column}
        \begin{column}{0.49\textwidth}
            \centering
            % TODO: provide actual figure file (e.g., figures/sw_gdp_growth.png)
            \includegraphics[width=\textwidth]{figures/SM_GDP_HD.png}
        \end{column}
    \end{columns}
\end{frame}

% ----------------------------------------------------------------------------
\begin{frame}{Example: Weekly Nowcasting}
    \vspace{-0.2em}
    \begin{block}{OECD Weekly Economic Tracker}
        A high-frequency ``weekly tracker'' uses weekly indicators to nowcast quarterly GDP growth in real time.
    \end{block}
    \begin{center}
        \includegraphics[width=0.92\textwidth]{figures/nowcasting_gdp.png}
    \end{center}
\end{frame}

% ----------------------------------------------------------------------------
\begin{frame}{Example: Estimating a Phillips Curve}
    Consider the reduced-form Phillips curve:
    \[
    \pi_t = \beta_0 + \beta_1 y_t + \beta_2 \pi_{t-1} + u_t
    \]
    where $\pi_t$ is inflation and $y_t$ is the output gap.
    
    \vspace{1em}
    
    \textbf{Questions of Interest:}
    \begin{itemize}
        \item What is the sign and size of $\beta_1$? (Inflation-output tradeoff)
        \item Is $\beta_2 \approx 1$? (Testing \textit{adaptive expectations}: $\E[\pi_t \mid \F_{t-1}] = \pi_{t-1}$)
    \end{itemize}
    
    \vspace{1em}
    
    \begin{block}{Key Insight}
        The presence of $\pi_{t-1}$ creates \textit{serial dependence} --- standard cross-sectional methods don't apply directly!
    \end{block}
\end{frame}

% ----------------------------------------------------------------------------
\begin{frame}{Two Types of Models}
    \begin{columns}
        \begin{column}{0.48\textwidth}
            \textbf{Structural Models}
            \begin{itemize}
                \item Behavioral functions with economic interpretation
                \item Causal relationships between variables
                \item Parameters have economic meaning
                \item Shocks are ``structural'' (e.g., monetary, fiscal)
            \end{itemize}
        \end{column}
        \begin{column}{0.48\textwidth}
            \textbf{Reduced-Form Models}
            \begin{itemize}
                \item Correlations between variables
                \item Descriptive analysis
                \item No direct economic interpretation
                \item Useful for \textit{forecasting}
            \end{itemize}
        \end{column}
    \end{columns}
    
    \vspace{1em}
    
    \begin{alertblock}{Trade-off}
        Structural models answer ``why?'' but require stronger assumptions.\\
        Reduced-form models answer ``what?'' and are useful for prediction.
    \end{alertblock}
\end{frame}

% ----------------------------------------------------------------------------
\begin{frame}{Univariate vs. Multivariate Models}
    \textbf{Univariate}: Series explained by its own past only
    \[
    \pi_t = \beta_0 + \beta_2 \pi_{t-1} + u_t
    \]
    
    \vspace{0.5em}
    
    \textbf{Multivariate}: Other variables also play a role
    \[
    \begin{pmatrix} 1 & -\beta_1 \\ 0 & 1 \end{pmatrix}
    \begin{pmatrix} \pi_t \\ y_t \end{pmatrix}
    = 
    \begin{pmatrix} \beta_0 \\ \alpha_0 \end{pmatrix}
    +
    \begin{pmatrix} \beta_2 & 0 \\ \alpha_1 & \alpha_2 \end{pmatrix}
    \begin{pmatrix} \pi_{t-1} \\ y_{t-1} \end{pmatrix}
    +
    \begin{pmatrix} u_{1,t} \\ u_{2,t} \end{pmatrix}
    \]
    
    \vspace{0.5em}
    
    \begin{itemize}
        \item Adding the output gap equation transforms it into a \textbf{Vector Autoregression (VAR)}
        \item Choice depends on the research question and data availability
    \end{itemize}
\end{frame}

% ----------------------------------------------------------------------------
\begin{frame}{Stylized Facts of Time Series}
    Time series data typically exhibit:
    
    \begin{enumerate}
        \item \textbf{Trend}: Long-term movement (growth, decline)
        \item \textbf{Seasonality}: Periodic and regular fluctuations (quarterly, monthly)
        \item \textbf{Cycles}: Medium-run fluctuations (business cycles) that are not strictly periodic (unlike seasonality)
        \item \textbf{Irregularity}: Erratic fluctuations (noise)
    \end{enumerate}
    
    \vspace{0.5em}
    
    \textbf{Additional Features:}
    \begin{itemize}
        \item \textbf{Heteroskedasticity}: Time-varying volatility (ARCH/GARCH effects)
        \item \textbf{Outliers}: Extreme observations
        \item \textbf{Structural breaks}: Changes in parameters over time
    \end{itemize}
    
    \vspace{0.5em}
    
    \begin{block}{Implication}
        Models must account for these features to properly represent the DGP.
    \end{block}
\end{frame}

% ============================================================================
% PART II: STOCHASTIC PROCESSES
% ============================================================================
\begin{frame}[plain]
    \begin{center}
        {\Large \textbf{Part II (a): Stochastic Processes}}\\[1em]
        {\large  Formal Definitions}
    \end{center}
\end{frame}

% ----------------------------------------------------------------------------
\begin{frame}{Why Formal Definitions?}
    \textbf{Goal}: Provide rigorous foundations for time series analysis
    
    \vspace{1em}
    
    \textbf{Key Questions:}
    \begin{itemize}
        \item What is a stochastic process mathematically?
        \item How do we relate the \textit{observed} time series to the \textit{underlying} process?
        \item Under what conditions can we learn from a single realization?
    \end{itemize}
    
    \vspace{1em}
    
    \textbf{Building Blocks:}
    \begin{enumerate}
        \item Probability spaces and measurability
        \item Filtrations (information evolving over time)
        \item Stochastic processes and their realizations
    \end{enumerate}
\end{frame}

% ----------------------------------------------------------------------------
\begin{frame}{Measurable Functions}
    \begin{definition}[Measurable Function]
        A function $f: \Omega \to \widetilde{\Omega}$ between measure spaces $(\Omega, \F)$ and $(\widetilde{\Omega}, \widetilde{\F})$ is $\F/\widetilde{\F}$-\textbf{measurable} if it preserves measurability:
        \[
        \forall B \in \widetilde{\F}, \quad f^{-1}(B) \in \F
        \]
        where $f^{-1}(B) = \{\omega \in \Omega : f(\omega) \in B\}$.
    \end{definition}
    
    \vspace{1em}
    
    \textbf{Intuition}: A measurable function ``respects'' the information structure. We can compute probabilities of events defined through $f$.
    
    \vspace{0.5em}
    
    \textbf{Why it matters}: Random variables must be measurable functions for probabilities to be well-defined.
\end{frame}

% ----------------------------------------------------------------------------
\begin{frame}{Filtered Probability Space}
    \begin{definition}[Filtered Probability Space]
        Let $(\Omega, \F, \P)$ be a probability space. A \textbf{filtration} is an increasing sequence of $\sigma$-algebras:
        \[
        \mathbb{F} := \{\F_t\}_{t \geq 0} \quad \text{with} \quad \F_0 \subseteq \F_1 \subseteq \F_2 \subseteq \cdots \subseteq \F
        \]
        The quadruple $(\Omega, \F, \mathbb{F}, \P)$ is a \textbf{filtered probability space}.
    \end{definition}
    
    \vspace{1em}
    
    \textbf{Intuition}: 
    \begin{itemize}
        \item $\F_t$ represents information available at time $t$
        \item The sequence is \textit{increasing}: questions answerable at time $t$ remain answerable at $t+1$
        \item Information grows over time (we don't forget!)
    \end{itemize}
\end{frame}

% ----------------------------------------------------------------------------
\begin{frame}{Measurability of Stochastic Processes}
    \begin{definition}[Adapted \& Predictable Processes]
        Let $(\Omega, \F, \mathbb{F}, \P)$ be a filtered probability space. A stochastic process $\{X_t\}_{t \geq 0}$ is:
        \begin{itemize}
            \item $\mathbb{F}$-\textbf{adapted} if $X_t$ is $\F_t$-measurable for every $t$
            \item $\mathbb{F}$-\textbf{predictable} if $X_t$ is $\F_{t-1}$-measurable for every $t$
        \end{itemize}
    \end{definition}
    
    \vspace{1em}
    
    \textbf{Interpretation}:
    \begin{itemize}
        \item \textbf{Adapted}: $X_t$ is known given information up to and including time $t$ (past \textit{or} present)
        \item \textbf{Predictable}: $X_t$ is known given information strictly before time $t$ (past only)
    \end{itemize}
    
    \vspace{0.5em}
    
    \begin{alertblock}{Example}
        Stock price $S_t$ is adapted (known at $t$), but not predictable (unknown at $t-1$).
    \end{alertblock}
\end{frame}

% ----------------------------------------------------------------------------
\begin{frame}{The Natural Filtration}
    \begin{definition}[$X$-adapted Filtration]
        Let $(\Omega, \F, \P)$ be a probability space and $\{X_t\}_{t \geq 0}$ a stochastic process. The \textbf{natural filtration} generated by $X$ is:
        \[
        \F_t^X = \sigma(X_0, X_1, \ldots, X_t)
        \]
        The process $X_t$ is $\F_t^X$-adapted by construction.
    \end{definition}
    
    \vspace{1em}
    
    \textbf{Intuition}: $\F_t^X$ is the information we can extract from observing the process up to time $t$.
    
    \vspace{0.5em}
    
    \textbf{Common notation}: $\F_{t-1}^X = \{X_{t-1}, X_{t-2}, \ldots\}$ is the ``information set'' at $t-1$.
\end{frame}

% ----------------------------------------------------------------------------
\begin{frame}{Stochastic Process: Definition}
    \begin{definition}[Stochastic Process]
        A \textbf{stochastic process} $\{X_t\}_{t \geq 0}$ on a filtered probability space $(\Omega, \F, \mathbb{F}, \P)$, which is $\mathbb{F}$-adapted, is an ordered sequence of random vectors:
        \[
        \{X_t\}_{t \geq 0} = \{\xbf_t(\omega) : \omega \in \Omega, \, t \in \mathcal{T}\}
        \]
        where $\xbf_t(\omega) \in \R^n$ for all $t \in \mathcal{T}$.
    \end{definition}
    
    \vspace{0.5em}
    
    \textbf{Two perspectives}:
    \begin{enumerate}
        \item \textbf{Fix $t$}: $\xbf_t(\cdot)$ is a random variable mapping $\Omega \to \R^n$
        \item \textbf{Fix $\omega$}: $\xbf_\cdot(\omega)$ is a deterministic function of time --- a \textbf{path} (or \textbf{realization})
    \end{enumerate}
\end{frame}

% ----------------------------------------------------------------------------
\begin{frame}{Time Series and Ensemble}
    \begin{definition}[Time Series]
        A \textbf{time series} $\{\xbf_t\}_{t=1}^T$ is a particular \textit{path} (realization) of the stochastic process $\{X_t\}_{t \geq 0}$, which is its \textbf{generating mechanism}.
    \end{definition}
    
    \vspace{0.5em}
    
    \begin{definition}[Ensemble]
        The set of all possible realizations of a stochastic process:
        \[
        \{\{\xbf_t(\omega) : t \in \mathcal{T}\} : \omega \in \Omega\}
        \]
        is called the \textbf{ensemble}.
    \end{definition}
    
    \vspace{0.5em}
    
    \begin{alertblock}{Key Insight}
        In macroeconometrics, we observe \textbf{one} realization from the ensemble. We cannot observe alternative histories!
    \end{alertblock}
\end{frame}

% ----------------------------------------------------------------------------
\begin{frame}{Time Series as a Realization of a Stochastic Process}
    \begin{center}
        \includegraphics[width=0.92\textwidth]{figures/realizations_sp.png}
    \end{center}


The five time series in the figure are realizations of the same stochastic process!

\end{frame}


% ============================================================================
% PART II.2: MODELS FOR STOCHASTIC PROCESSES
% ============================================================================
\begin{frame}[plain]
    \begin{center}
        {\Large \textbf{Part II (b): Models for Stochastic Processes}}\\[1em]
        {\large DGP, Stationarity, and Structural/Reduced-Form Models}
    \end{center}
\end{frame}

% ----------------------------------------------------------------------------
\begin{frame}{Data-Generating Process (DGP)}
    \begin{definition}[Data-Generating Process]
        The \textbf{DGP} of $\{X_t\}_{t \geq 0}$ is the probabilistic rule that generates the process over time. A convenient characterization is given by:
        \[
        F(X_0)
        \quad \text{and} \quad
        \{\,F(X_t \mid \F_{t-1}^X)\,\}_{t \geq 1},
        \]
        i.e., the initial distribution and the collection of one-step-ahead conditional distributions given the information set.
    \end{definition}
    
    \vspace{1em}
    
    \textbf{Problem}: This is an \textit{infinite-dimensional} object!
    \begin{itemize}
        \item Analytically intractable
        \item Cannot estimate infinitely many parameters
    \end{itemize}
    
    \vspace{0.5em}
    
    \textbf{Question}: Can we represent it by something finite-dimensional?
\end{frame}

% ----------------------------------------------------------------------------
\begin{frame}{Finite-Dimensional Distributions}
    \begin{definition}[Finite-Dimensional Distribution]
        Let $\{t_1, t_2, \ldots, t_s\} \subset \mathcal{T}$ be a finite set. The joint distribution:
        \[
        F_{t_1, \ldots, t_s}(b_1, \ldots, b_s) = \P(\xbf_{t_1} \leq b_1, \ldots, \xbf_{t_s} \leq b_s)
        \]
        The family of all such distributions is the \textbf{finite-dimensional distribution} of the process.
    \end{definition}
    
    \vspace{0.5em}
    
    \begin{theorem}
        Under general conditions, the probabilistic structure of $\{X_t\}_{t \geq 0}$ is \textbf{completely specified} by its finite-dimensional distributions for all $s$ and all choices of $\{t_1, \ldots, t_s\}$.
    \end{theorem}
    
    \vspace{0.5em}
    
    \textbf{Implication}: We can work with finite-dimensional objects!
\end{frame}

% ----------------------------------------------------------------------------
\begin{frame}{Time Series Model}
    \begin{definition}[Time Series Model]
        An econometric time series model is a family of functions:
        \[
        \{\mathcal{M}(\xbf_t, \xbf_{t-1}, \ldots, d_t; \psi) : \psi \in \Psi \subseteq \R^p\}
        \]
        that aims to represent the true DGP, where:
        \begin{itemize}
            \item $\psi$ is a $p$-dimensional parameter vector
            \item $d_t$ represents possible structural breaks
        \end{itemize}
    \end{definition}
    
    \vspace{0.5em}
    
    \textbf{Goal}: Estimate $\hat{\psi}$ based on assumptions about $\mathcal{M}$.
    
    \vspace{0.5em}
    
    \textbf{Two flavors}:
    \begin{itemize}
        \item Model for the \textit{distribution}: $\mathcal{M}_D$
        \item Model for the \textit{conditional expectation}: $\mathcal{M}_E$ (most common)
    \end{itemize}
\end{frame}

% ----------------------------------------------------------------------------
\begin{frame}{Correct Specification Axioms}
    \begin{block}{Axiom: Correct Specification for Distribution}
        \[
        \mathcal{M}_D(\xbf_t, \xbf_{t-1}, \ldots, d_t; \psi) = F(X_t \mid \F_{t-1}^X)
        \]
    \end{block}
    
    \begin{block}{Axiom: Correct Specification for Expectation}
        \[
        \mathcal{M}_E(\xbf_t, \xbf_{t-1}, \ldots, d_t; \psi) = \E[\xbf_t | \F_{t-1}^X]
        \]
    \end{block}
    
    \vspace{0.5em}
    
    \textbf{In practice}: We usually model only the conditional expectation
    \begin{itemize}
        \item Similar to CEF approach in cross-sectional econometrics
        \item Reduces dimensionality dramatically
    \end{itemize}
\end{frame}

% ----------------------------------------------------------------------------
\begin{frame}{Structural Model}
    \begin{definition}[Structural Model]
        Given data $\zbf = (\ybf, \xbf)$ from the joint process $\{Y_t, X_t\}_{t \geq 0}$, a \textbf{structural model} for the conditional expectation is:
        \[
        g(\ybf, \xbf, \mathbf{d}, \ubf; \psi_0) = 0
        \]
        where $\ubf$ are \textbf{structural shocks} and $\mathbf{d}$ are structural breaks.
    \end{definition}
    
    \vspace{1em}
    
    \textbf{Key Features}:
    \begin{itemize}
        \item Shocks $\ubf$ have \textit{economic interpretation} (monetary, fiscal, technology, \ldots)
        \item Parameters $\psi_0$ have \textit{economic meaning} (elasticities, policy parameters, \ldots)
        \item Implies \textit{causal} relationships
    \end{itemize}
\end{frame}

% ----------------------------------------------------------------------------
\begin{frame}{Reduced-Form Model}
    \begin{definition}[Reduced-Form Model]
        If $\{Y_t\}_{t \geq 0}$ has a unique solution as a function of past values and exogenous variables:
        \[
        \ybf_t = h(\ybf_{t-1}, \ybf_{t-2}, \ldots, \xbf_t, \vbf_t, \mathbf{d}_t; \theta_0)
        \]
        where $\theta_0 = m(\psi_0)$ and $\vbf_t = f(\ubf_t)$ are \textbf{reduced-form} parameters and shocks.
    \end{definition}
    
    \vspace{0.5em}
    
    \textbf{Key Features}:
    \begin{itemize}
        \item Parameters and shocks \textit{lose} economic interpretation
        \item Useful for \textbf{forecasting}: $\E[\ybf_t | \xbf_t, \F_{t-1}]$ is the best MSE predictor
        \item Easier to estimate (no simultaneity issues)
    \end{itemize}
    
    \vspace{0.5em}
    
    \begin{alertblock}{Trade-off}
        Reduced form: good for prediction, bad for policy analysis (Lucas critique!)
    \end{alertblock}
\end{frame}

% ----------------------------------------------------------------------------
\begin{frame}{The Incidental Parameter Problem}
    \textbf{Without restrictions}: Suppose $Y_t \sim \mathcal{N}(\mu_t, \Gamma_{ts})$
    
    \vspace{0.5em}
    
    We need to estimate:
    \[
    \E(Y^T) = \begin{bmatrix} \mu_1' & \mu_2' & \cdots & \mu_T' \end{bmatrix}'
    \]
    \[
    \Var(Y^T) = \begin{bmatrix}
        \Gamma_{11} & \Gamma_{12} & \cdots & \Gamma_{1T} \\
        \Gamma_{21} & \Gamma_{22} & \cdots & \Gamma_{2T} \\
        \vdots & \vdots & \ddots & \vdots \\
        \Gamma_{T1} & \Gamma_{T2} & \cdots & \Gamma_{TT}
    \end{bmatrix}
    \]
    
    \vspace{0.5em}
    
    \begin{alertblock}{Problem}
        Number of parameters grows with $T$ $\Rightarrow$ LLN and CLT fail!
    \end{alertblock}
\end{frame}

% ----------------------------------------------------------------------------
\begin{frame}{Solution: Restrictions on the Process}
    \textbf{Two types of restrictions}:
    
    \vspace{0.5em}
    
    \begin{enumerate}
        \item \textbf{Time-homogeneity} (Stationarity)
        \begin{itemize}
            \item All observations come from distributions with time-invariant features
            \item Reduces: $\mu_t \to \mu$, $\Gamma_{ts} \to \Gamma_{|t-s|}$
        \end{itemize}
        
        \vspace{1em}
        
        \item \textbf{Memory restrictions} (Ergodicity/Mixing)
        \begin{itemize}
            \item Each new observation contains ``new information''
            \item Enables consistent estimation from a single realization
        \end{itemize}
    \end{enumerate}
    
    \vspace{1em}
    
    \textbf{With stationarity}:
    \[
    \E(Y^T) = \begin{bmatrix} \mu' & \mu' & \cdots & \mu' \end{bmatrix}', \quad
    \Var(Y^T) = \begin{bmatrix}
        \Gamma_0 & \Gamma_1 & \cdots & \Gamma_{T-1} \\
        \Gamma_{-1} & \Gamma_0 & \cdots & \Gamma_{T-2} \\
        \vdots & \vdots & \ddots & \vdots
    \end{bmatrix}
    \]
\end{frame}

% ----------------------------------------------------------------------------
\begin{frame}{Weak Stationarity}
    \begin{definition}[Weak Stationarity]
        A stochastic process $\{Y_t\}$ is \textbf{weakly stationary} (covariance stationary, second-order stationary) if:
        \begin{enumerate}
            \item $\E[Y_t] = \mu$ for all $t \in \mathcal{T}$, where $\|\mu\| < \infty$
            \item $\E[(Y_t - \mu)(Y_{t-h} - \mu)'] = \Gamma_h$ for all $t, h$, where $\|\Gamma_h\| < \infty$
        \end{enumerate}
    \end{definition}
    
    \vspace{0.5em}
    
    \textbf{Key Properties}:
    \begin{itemize}
        \item Mean is constant over time
        \item Autocovariance depends only on the \textit{lag} $h$, not on $t$
        \item Variance $\Gamma_0$ is constant
    \end{itemize}
    
    \vspace{0.5em}
    
    \begin{lemma}
        For weakly stationary processes: $\Gamma_h = \Gamma_{-h}'$
    \end{lemma}
\end{frame}

% ----------------------------------------------------------------------------
\begin{frame}{Example: Autocovariance Matrices}
    For a bivariate process $Y_t = (Y_{1t}, Y_{2t})'$:
    
    \[
    \Gamma_0 = \Var(Y_t) = \begin{bmatrix}
        \Var(Y_{1t}) & \Cov(Y_{1t}, Y_{2t}) \\
        \Cov(Y_{1t}, Y_{2t}) & \Var(Y_{2t})
    \end{bmatrix}
    \]
    
    \[
    \Gamma_1 = \Cov(Y_t, Y_{t-1}) = \begin{bmatrix}
        \Cov(Y_{1t}, Y_{1,t-1}) & \Cov(Y_{1t}, Y_{2,t-1}) \\
        \Cov(Y_{2t}, Y_{1,t-1}) & \Cov(Y_{2t}, Y_{2,t-1})
    \end{bmatrix}
    \]
    
    \vspace{0.5em}
    
    \textbf{Interpretation}:
    \begin{itemize}
        \item Diagonal: \textit{auto}covariances of each component
        \item Off-diagonal: \textit{cross}-covariances between components
    \end{itemize}
\end{frame}

% ----------------------------------------------------------------------------
\begin{frame}{Strong Stationarity}
    \begin{definition}[Strong Stationarity]
        A stochastic process $\{Y_t\}$ is \textbf{strongly stationary} (strictly stationary) if all finite-dimensional distributions are invariant to time shifts: for any $n \in \N$, any times $t_1,\ldots,t_n$, any shift $h \in \Z$, and any $y_1,\ldots,y_n$,
        \[
        F_{t_1,\ldots,t_n}(y_1,\ldots,y_n) = F_{t_1+h,\ldots,t_n+h}(y_1,\ldots,y_n).
        \]
    \end{definition}
    
    % \vspace{1em}
    
    \textbf{Relationship to Weak Stationarity}:
    \begin{itemize}
        \item Strong stationarity $\Rightarrow$ Weak stationarity \textit{if moments exist}
        \item Weak stationarity $\not\Rightarrow$ Strong stationarity (in general)
        \item For \textbf{Gaussian} processes: Weak $\Leftrightarrow$ Strong
    \end{itemize}
    
    % \vspace{0.5em}
    
    \begin{alertblock}{Cauchy Counterexample}
        A Cauchy-distributed process can be strongly stationary but \textit{not} weakly stationary (moments undefined!).
    \end{alertblock}
\end{frame}

% ============================================================================
% PART II.4: LINEAR OPERATORS
% ============================================================================
\begin{frame}[plain]
    \begin{center}
        {\Large \textbf{Part II (c): Operations on Stochastic Processes}}\\[1em]
        {\large Linear Operators}
    \end{center}
\end{frame}

% ----------------------------------------------------------------------------
\begin{frame}{Linear Operators}
    \begin{definition}[Linear Operator]
        Let $X$ and $Y$ be vector spaces over field $F$. The operator $T: X \to Y$ is \textbf{linear} if:
        \[
        T(\alpha x_1 + \beta x_2) = \alpha T(x_1) + \beta T(x_2)
        \]
        for all $x_1, x_2 \in X$ and $\alpha, \beta \in F$.
    \end{definition}
    
    \vspace{1em}
    
    \textbf{Context}: Time series form a vector space. We can define useful linear operators on this space.
    
    \vspace{0.5em}
    
    \textbf{Key operators}:
    \begin{itemize}
        \item Lag operator $L$
        \item Difference operator $\Delta$
    \end{itemize}
\end{frame}

% ----------------------------------------------------------------------------
\begin{frame}{The Lag Operator}
    % \vspace{-2em}
    \begin{definition}[Lag Operator]
        For a stochastic process $\{\zbf_t\}$, the \textbf{lag operator} $L$ is defined by:
        \begin{align*}
            % L\zbf_t &= \zbf_{t-1} \\
            L^j \zbf_t &= \zbf_{t-j} \quad \forall j \in \N
        \end{align*}
    \end{definition}
    
    % \vspace{0.5em}
    
    \textbf{Properties}:
    \begin{itemize}
        \item $L$ is a linear operator
        \item $L^0 = I$ (identity)
        \item Powers compose: $L^j L^k = L^{j+k}$
    \end{itemize}
    
    % \vspace{0.5em}
    
    \begin{proposition}
        For $|\alpha| < 1$:
        \[
        (1 - \alpha L)^{-1} = 1 + \alpha L + \alpha^2 L^2 + \cdots = \sum_{j=0}^{\infty} \alpha^j L^j
        \]
    \end{proposition}
\end{frame}

% ----------------------------------------------------------------------------
\begin{frame}{The Difference Operator}
    \begin{definition}[Difference Operator]
        The \textbf{difference operator} $\Delta$ is defined by:
        \begin{align*}
            \Delta \zbf_t &= (I - L)\zbf_t = \zbf_t - \zbf_{t-1} \\
            \Delta^j \zbf_t &= (I - L)^j \zbf_t \quad \forall j \in \N_+ \\
            \Delta_j \zbf_t &= (I - L^j)\zbf_t = \zbf_t - \zbf_{t-j}
        \end{align*}
    \end{definition}
    
    \vspace{0.5em}
    
    \textbf{Common uses}:
    \begin{itemize}
        \item $\Delta y_t$: First difference (removes linear trend)
        \item $\Delta^2 y_t$: Second difference (removes quadratic trend)
        \item $\Delta_{12} y_t = y_t - y_{t-12}$: Seasonal difference (monthly data)
    \end{itemize}
    
    \vspace{0.5em}
    
    \textbf{Key insight}: Differencing can transform non-stationary series into stationary ones (unit root processes).
\end{frame}

% ----------------------------------------------------------------------------
\begin{frame}{Lag Polynomials}
    \textbf{Notation}: We can write models compactly using \textbf{lag polynomials}:
    
    \[
    \Phi(L) = 1 - \phi_1 L - \phi_2 L^2 - \cdots - \phi_p L^p
    \]
    
    \vspace{0.5em}
    
    \textbf{Example}: AR($p$) model
    \[
    y_t = \phi_1 y_{t-1} + \cdots + \phi_p y_{t-p} + u_t
    \]
    becomes
    \[
    \Phi(L) y_t = u_t
    \]
    
    \vspace{0.5em}
    
    \textbf{Inversion}: If roots of $\Phi(z) = 0$ are outside the unit circle:
    \[
    y_t = \Phi(L)^{-1} u_t = \sum_{j=0}^{\infty} \psi_j u_{t-j}
    \]
    This is the \textbf{MA($\infty$) representation}.
\end{frame}

% ============================================================================
% PART III: EXAMPLES
% ============================================================================
\begin{frame}[plain]
    \begin{center}
        {\Large \textbf{Part III: Examples}}\\[1em]
        {\large Applications to Macroeconomics}
    \end{center}
\end{frame}

% ----------------------------------------------------------------------------
\begin{frame}{Why Examples Matter}
    \textbf{Goal}: Connect abstract definitions to concrete economic models
    
    \vspace{1em}
    
    \textbf{Key insight}: Many structural macro models admit reduced-form time series representations
    
    \vspace{1em}
    
    \textbf{Examples we'll cover}:
    \begin{enumerate}
        \item Monetary policy transmission (Phillips curve + Taylor rule)
        \item Samuelson's multiplier-accelerator (1939)
        \item Intertemporal optimization (Euler equations)
    \end{enumerate}
    
    \vspace{1em}
    
    Each shows: \textbf{Structural model} $\to$ \textbf{Reduced form} (AR, VAR, ARMA)
\end{frame}

% ----------------------------------------------------------------------------
\begin{frame}{Example 1: Monetary Policy Transmission}
    \textbf{Structural model}:
    \begin{align*}
        \pi_t &= \lambda y_t + \pi_t^e + u_{1t} & \text{(Phillips curve)} \\
        y_t &= \gamma(i_{t-1} - \pi_t^e) + u_{2t} & \text{(IS curve)} \\
        \pi_t^e &= \pi_{t-1} & \text{(Adaptive expectations)} \\
        i_t &= i^* + \rho(\pi_t - \pi^*) & \text{(Taylor rule)}
    \end{align*}
    where:
    \begin{itemize}
        \item $\pi_t$: inflation, $y_t$: output gap, $i_t$: interest rate
        \item $\lambda \in (0,1)$, $\gamma \in (-1, 0)$, $\rho \geq 0$
        \item $\ubf_t = (u_{1t}, u_{2t})' \sim \text{NID}(\mathbf{0}, \Omega)$
    \end{itemize}
\end{frame}

% ----------------------------------------------------------------------------
\begin{frame}{Monetary Policy: Reduced Form}
    \textbf{Substituting} the Taylor rule and expectations:
    \begin{align*}
        \pi_t &= \lambda y_t + \pi_{t-1} + u_{1t} \\
        y_t &= \gamma(i^* - \rho\pi^*) + \gamma(\rho - 1)\pi_{t-1} + u_{2t}
    \end{align*}
    
    \textbf{In matrix form} (VAR(1)):
    \[
    \underbrace{\begin{bmatrix} 1 & -\lambda \\ 0 & 1 \end{bmatrix}}_{B}
    \begin{bmatrix} \pi_t \\ y_t \end{bmatrix}
    = 
    \begin{bmatrix} 0 \\ \gamma(i^* - \rho\pi^*) \end{bmatrix}
    +
    \begin{bmatrix} 1 & 0 \\ \gamma(\rho-1) & 0 \end{bmatrix}
    \begin{bmatrix} \pi_{t-1} \\ y_{t-1} \end{bmatrix}
    +
    \begin{bmatrix} u_{1t} \\ u_{2t} \end{bmatrix}
    \]
    
    Since $B$ is invertible, multiply by $B^{-1}$:
    \[
    \mathbf{w}_t = \mathbf{c}_0 + C_1 \mathbf{w}_{t-1} + \mathbf{v}_t
    \]
    where $\mathbf{v}_t = B^{-1}\ubf_t$ are \textit{reduced-form} shocks.
\end{frame}

% ----------------------------------------------------------------------------
\begin{frame}{Monetary Policy: Univariate Representation}
    \textbf{Isolating inflation} (since $y_{t-1}$ doesn't appear in $\pi_t$ equation):
    \[
    \pi_t = \lambda\gamma(i^* - \rho\pi^*) + [1 + \lambda\gamma(\rho - 1)]\pi_{t-1} + u_{1t} + \lambda u_{2t}
    \]
    
    This is an \textbf{AR(1)} for inflation:
    \[
    \pi_t = \phi_0 + \phi_1 \pi_{t-1} + v_{1t}
    \]
    
    \vspace{1em}
    
    \textbf{Key observations}:
    \begin{itemize}
        \item Same DGP, different information sets
        \item VAR uses $({\pi_{t-1}, y_{t-1}})$; AR(1) uses only $\pi_{t-1}$
        \item Structural shocks $\ubf_t$ become reduced-form shocks $\mathbf{v}_t$
    \end{itemize}
\end{frame}

% ----------------------------------------------------------------------------
\begin{frame}{Example 2: Samuelson's Multiplier-Accelerator (1939)}
    \textbf{Historical context}: First mathematical model after Keynes' \textit{General Theory}
    
    \vspace{0.5em}
    
    \textbf{Structural model}:
    \begin{align*}
        y_t &= c_t + i_t + g & \text{(National income identity)} \\
        c_t &= \alpha y_{t-1} + \varepsilon_t & \text{(Consumption function)} \\
        i_t &= \beta(c_t - c_{t-1}) & \text{(Accelerator)}
    \end{align*}
    
    \vspace{0.5em}
    
    \textbf{Reduced form}: Substitute to get
    \begin{align*}
        y_t &= g + \alpha(1+\beta)y_{t-1} - \beta\alpha y_{t-2} - \beta\varepsilon_{t-1} + (1+\beta)\varepsilon_t
    \end{align*}
    
    This is an \textbf{ARMA(2,1)}:
    \[
    y_t = \phi_0 + \phi_1 y_{t-1} + \phi_2 y_{t-2} + \theta u_{t-1} + u_t
    \]
    where $u_t = (1+\beta)\varepsilon_t$ and $\theta = -\beta/(1+\beta)$.
\end{frame}

% ----------------------------------------------------------------------------
\begin{frame}{Example 3: Euler Equation from Optimization}
    \textbf{Consumer's problem}:
    \[
    \max_{c_{t+i}, A_{t+i}} \E\left[\sum_{i=0}^{\infty} (1+\delta)^{-i} U(c_{t+i}) \Big| \F_t\right]
    \]
    subject to budget constraint and transversality condition.
    
    \vspace{0.5em}
    
    \textbf{With CRRA utility} $U(c) = \frac{c^{1-\gamma}}{1-\gamma}$:
    
    \vspace{0.5em}
    
    \textbf{Euler equation}:
    \[
    \E\left[\frac{1+r}{1+\delta} c_{t+1}^{-\gamma} - c_t^{-\gamma} \Big| \F_t\right] = 0
    \]
    
    \vspace{0.5em}
    
    \textbf{Parameters of interest}:
    \begin{itemize}
        \item $\delta$: Discount rate (patience)
        \item $\gamma$: Risk aversion (intertemporal substitution)
    \end{itemize}
    
    This is a \textit{conditional moment restriction} --- basis for GMM estimation!
\end{frame}

\begin{frame}{Example 3: Structural to Reduced Form}
    \begin{itemize}
    \item We can map the structural to a reduced form model under assumptions:
    \begin{enumerate}
        \item \textbf{Assumption (1)}: $\delta  = r $
        \item \textbf{Assumption (2)}: $y_t = \alpha y_{t-1} + \varepsilon_t$
    \end{enumerate}
    \vspace{0.5em}
    \item Under Assumption (1) and (2),
    \begin{align*}
        c_t = r(1+r) A_{t-1} + r \frac{1+r}{1+r - \alpha}y_t + \varepsilon_t
    \end{align*}
    where $A_{t-1}$ is beginning of period assets, and $\varepsilon_t$ is an autocorrelated error
    \item The above equation is a reduced form model for consumption
    \end{itemize}
\end{frame}
% ----------------------------------------------------------------------------
\begin{frame}{Common Patterns}
    \textbf{What we've seen}:
    
    \vspace{0.5em}
    
    \begin{center}
    \begin{tabular}{lll}
        \toprule
        \textbf{Example} & \textbf{Structural} & \textbf{Reduced Form} \\
        \midrule
        Monetary policy & 4 equations & VAR(1) or AR(1) \\
        Samuelson & 3 equations & ARMA(2,1) \\
        Euler equation & FOC & Moment restriction \\
        \bottomrule
    \end{tabular}
    \end{center}
    
    \vspace{1em}
    
    \textbf{Key insight}: Economic models often have \textit{autoregressive} reduced forms
    \begin{itemize}
        \item AR/MA/ARMA capture stylized facts: persistence, mean reversion, cycles
        \item VARs capture multivariate dynamics
        \item Euler equations lead to GMM/rational expectations estimation
    \end{itemize}
\end{frame}

% ----------------------------------------------------------------------------
\begin{frame}{Summary}
    \textbf{Today we covered}:
    
    \vspace{0.5em}
    
    \begin{enumerate}
        \item \textbf{Introduction}
        \begin{itemize}
            \item Time series: dynamic relationships + forecasting
            \item Structural vs. reduced-form models
            \item Stylized facts
        \end{itemize}
        
        \vspace{0.3em}
        
        \item \textbf{Stochastic Processes}
        \begin{itemize}
            \item Filtrations, measurability, DGP
            \item Stationarity (weak, strong) --- reduces parameters
            \item Lag and difference operators
        \end{itemize}
        
        \vspace{0.3em}
        
        \item \textbf{Examples}
        \begin{itemize}
            \item Monetary transmission $\to$ VAR
            \item Samuelson $\to$ ARMA
            \item Euler equation $\to$ GMM
        \end{itemize}
    \end{enumerate}
    
    \vspace{0.5em}
    
    \textit{Appendix: Ergodicity \& Mixing, Martingales}
\end{frame}

% ----------------------------------------------------------------------------
\begin{frame}{Preview: What's Next}
    \textbf{Section 7.4}: Reduced-Form Models in Detail
    \begin{itemize}
        \item AR($p$): Definition, stationarity, estimation
        \item MA($q$): Definition, invertibility
        \item ARMA($p,q$): Combining both
    \end{itemize}
    
    \vspace{0.5em}
    
    \textbf{Section 7.5}: Vector Autoregressions
    \begin{itemize}
        \item VAR($p$) estimation and inference
        \item Impulse response functions
        \item Structural identification
    \end{itemize}
    
    \vspace{0.5em}
    
    \textbf{Section 7.6}: Estimation Methods
    \begin{itemize}
        \item OLS for AR models
        \item Maximum likelihood
        \item GMM for Euler equations
    \end{itemize}
\end{frame}

% ============================================================================
% APPENDIX
% ============================================================================
\appendix

\begin{frame}[plain]
    \begin{center}
        {\Large \textbf{Appendix}}
    \end{center}
\end{frame}

% ============================================================================
% APPENDIX: ERGODICITY AND MIXING (moved from main presentation)
% ============================================================================
\begin{frame}[plain]
    \begin{center}
        {\Large \textbf{Appendix: Ergodicity \& Mixing}}\\[1em]
        {\large Section 7.2.2 (cont.)}
    \end{center}
\end{frame}

% ----------------------------------------------------------------------------
\begin{frame}{$\alpha$-Mixing Coefficient}
    \begin{definition}[$\alpha$-Mixing Coefficient]
        Let $\mathcal{G}$ and $\mathcal{H}$ be $\sigma$-subfields of $\F$. The $\alpha$-\textbf{mixing coefficient} is:
        \[
        \alpha(\mathcal{G}, \mathcal{H}) = \sup_{G \in \mathcal{G}, H \in \mathcal{H}} |\P(G \cap H) - \P(G)\P(H)|
        \]
    \end{definition}
    
    \vspace{0.5em}
    
    \textbf{Intuition}: Measures how ``far from independent'' two $\sigma$-algebras are.
    \begin{itemize}
        \item $\alpha = 0$: $\mathcal{G}$ and $\mathcal{H}$ are independent
        \item $\alpha > 0$: Some dependence exists
    \end{itemize}
    
    \vspace{0.5em}
    
    \textbf{In time series context}:
    \[
    \mathcal{G} = \F_{-\infty}^0 = \sigma(\ldots, X_{-1}, X_0), \quad
    \mathcal{H} = \F_m^{+\infty} = \sigma(X_m, X_{m+1}, \ldots)
    \]
    $\mathcal{G}$ = ``past'', $\mathcal{H}$ = ``distant future''
\end{frame}

% ----------------------------------------------------------------------------
\begin{frame}{$\alpha$-Mixing Condition}
    \begin{definition}[$\alpha$-Mixing Condition]
        A sequence $\{X_t\}_{t=-\infty}^{\infty}$ is $\alpha$-\textbf{mixing} (strong mixing) if:
        \[
        \lim_{m \to \infty} \alpha_m = 0
        \]
        where
        \[
        \alpha_m = \sup_t \alpha(\F_{-\infty}^t, \F_{t+m}^{\infty})
        \]
    \end{definition}
    
    \vspace{1em}
    
    \textbf{Intuition}: As the gap $m$ increases, past and future become \textit{asymptotically independent}.
    
    \vspace{0.5em}
    
    \begin{corollary}[For Gaussian Random Variables]
        A useful mixing condition is \textbf{absolute summability}:
        \[
        \sum_{k=-\infty}^{\infty} |\Gamma_k^{ij}| < \infty \quad \forall\, i, j
        \]
        Autocovariances must decay to zero fast enough.
    \end{corollary}
\end{frame}

% ----------------------------------------------------------------------------
\begin{frame}{Transformations and Invariant Events}
    \begin{definition}[Transformation]
        A \textbf{transformation} $T: \Omega \to \Omega$ is a 1-1 measurable mapping of outcomes.
    \end{definition}
    
    \begin{definition}[Measure-Preserving Transformation]
        $T$ is \textbf{measure-preserving} if $\P(TE) = \P(E)$ for all $E \in \F$.
    \end{definition}
    
    \vspace{0.5em}
    
    \textbf{Example: Shift Transformation}
    
    The \textbf{back-shift operator} $T$: $X_t(T\omega) = X_{t+1}(\omega)$
    
    \vspace{0.5em}
    
    \begin{center}
    \begin{tabular}{l}
        $\omega = (\ldots, x_1, x_2, x_3, \ldots)$ \\
        $T\omega = (\ldots, x_2, x_3, x_4, \ldots)$ \\
        $T^2\omega = (\ldots, x_3, x_4, x_5, \ldots)$
    \end{tabular}
    \end{center}
    
    Under strict stationarity, the shift transformation is measure-preserving.
\end{frame}

% ----------------------------------------------------------------------------
\begin{frame}{Ergodicity}
    \begin{definition}[Ergodicity]
        A strictly stationary sequence $\{X_t(\omega)\}$ is \textbf{ergodic} if the probability of every invariant event is either 0 or 1.
    \end{definition}
    
    \vspace{0.5em}
    
    \textbf{Invariant event}: $E$ such that $TE = E$ (unchanged by time shift).
    
    \vspace{0.5em}
    
    \textbf{Intuition}: If an invariant event has probability strictly between 0 and 1, there exist ``parallel universes'' that never communicate with each other.
    
    \vspace{1em}
    
    \begin{theorem}[Ergodic Theorem]
        Let $\{X_t\}$ be weakly stationary and ergodic with $\E[|X_t|] < \infty$. Then:
        \[
        \frac{1}{T} \sum_{t=1}^T X_t(\omega) \xrightarrow{a.s.} \E[X_t]
        \]
    \end{theorem}
    
    \textbf{This is a Strong LLN for dependent processes!}
\end{frame}

% ----------------------------------------------------------------------------
\begin{frame}{Example: Ergodicity Failure (I)}
    \textbf{Example}: Simple two-sequence universe
    
    \vspace{0.5em}
    
    Let $\Omega = \{\omega_1, \omega_2\}$ where:
    \[
    \omega_1 = (\ldots, 1, 1, 1, \ldots), \quad \omega_2 = (\ldots, 0, 0, 0, \ldots)
    \]
    with $\P(\omega_1) = p \in (0, 1)$ and $\P(\omega_2) = 1 - p$.
    
    \vspace{0.5em}
    
    Define $X_t(\omega_1) = 1$ and $X_t(\omega_2) = 0$ for all $t$.
    
    \vspace{1em}
    
    \textbf{Analysis}:
    \begin{itemize}
        \item Both $\omega_1$ and $\omega_2$ are invariant under the shift $T$
        \item $\P(\omega_1) = p \in (0, 1)$ --- not 0 or 1!
        \item $\Rightarrow$ The process is \textbf{not ergodic}
    \end{itemize}
    
    \vspace{0.5em}
    
    If we happen to draw $\omega_1$, we see only 1's forever and learn nothing about $\omega_2$.
\end{frame}

% ----------------------------------------------------------------------------
\begin{frame}{Example: Ergodicity Failure (II)}
    \textbf{Example}: Random intercept
    \[
    X_t = U_t + Z, \quad t \in \Z
    \]
    where $\{U_t\} \sim \text{i.i.d. Uniform}[0,1]$ and $Z \sim N(0,1)$, independent of $U_t$.
    
    \vspace{0.5em}
    
    \textbf{What happens?}
    \[
    \bar{X}_n = \frac{1}{n}\sum_{t=1}^n X_t = \bar{U}_n + Z \xrightarrow{a.s.} \frac{1}{2} + Z
    \]
    
    The sample mean converges to a \textit{random variable}, not a constant!
    
    \vspace{0.5em}
    
    \textbf{Why?} The autocovariance:
    \[
    \Cov(X_t, X_{t+h}) = \Var(Z) = 1 \quad \forall h \neq 0
    \]
    Dependence is \textit{too strong} --- it never dies out!
    
    \vspace{0.5em}
    
    \begin{alertblock}{Lesson}
        Without ergodicity, time averages $\neq$ population averages.
    \end{alertblock}
\end{frame}

% ============================================================================
% APPENDIX: MARTINGALES (moved from main presentation)
% ============================================================================
\begin{frame}[plain]
    \begin{center}
        {\Large \textbf{Appendix: Martingales}}\\[1em]
        {\large Section 7.2.4}
    \end{center}
\end{frame}

% ----------------------------------------------------------------------------
\begin{frame}{Doob-Dynkin Lemma}
    \begin{lemma}[Doob-Dynkin]
        Let $X: \Omega \to \R^n$ and $Y: \Omega \to \R^m$ be random vectors. Then $Y$ is $\F^X$-measurable if and only if there exists a measurable function $g: \R^n \to \R^m$ such that:
        \[
        Y = g \circ X
        \]
    \end{lemma}
    
    \vspace{1em}
    
    \begin{corollary}
        We always have:
        \[
        \E[Y | \F^X] = \E[Y | X] = g \circ X
        \]
        for some measurable $g$.
    \end{corollary}
    
    \vspace{0.5em}
    
    \textbf{Intuition}: Conditional expectations are functions of the conditioning variables.
\end{frame}

% ----------------------------------------------------------------------------
\begin{frame}{Martingale Process}
    \begin{definition}[Martingale]
        A sequence $\{Y_t\}_{t \in \N}$ is a \textbf{martingale} w.r.t. $\{\F_t\}$ if:
        \begin{enumerate}
            \item $\E[\|Y_t\|] < \infty$ for all $t$
            \item $Y_t$ is $\F_t$-measurable
            \item $\E[Y_t | \F_{t-1}] = Y_{t-1}$ for all $t$
        \end{enumerate}
    \end{definition}
    
    \vspace{1em}
    
    \textbf{Intuition}: The best forecast of tomorrow's value is today's value.
    
    \vspace{0.5em}
    
    \textbf{Examples}:
    \begin{itemize}
        \item Random walk: $S_t = S_{t-1} + \varepsilon_t$ with $\E[\varepsilon_t | \F_{t-1}] = 0$
        \item Stock prices (under risk-neutral measure)
        \item Cumulative sum of fair bets
    \end{itemize}
\end{frame}

% ----------------------------------------------------------------------------
\begin{frame}{Martingale Difference Sequence}
    \begin{definition}[Martingale Difference Sequence (MDS)]
        A sequence $\{Y_t\}_{t \in \N}$ is a \textbf{martingale difference sequence} w.r.t. $\{\F_t\}$ if:
        \begin{enumerate}
            \item $\E[\|Y_t\|] < \infty$ for all $t$
            \item $Y_t$ is $\F_t$-measurable
            \item $\E[Y_t | \F_{t-1}] = 0$ for all $t$
        \end{enumerate}
    \end{definition}
    
    \vspace{0.5em}
    
    \textbf{Key property}: MDS are \textbf{uncorrelated} over time!
    
    \begin{corollary}
        If $\{Y_t\}$ is MDS, then $\E[Y_t Y_{t-j}] = 0$ for all $j \neq 0$.
    \end{corollary}
    
    \vspace{0.5em}
    
    \textbf{Proof}: $\E[Y_t Y_{t-j}] = \E[\E[Y_t | \F_{t-1}] Y_{t-j}] = \E[0 \cdot Y_{t-j}] = 0$
    
    \vspace{0.5em}
    
    \textbf{Note}: MDS is \textit{weaker} than i.i.d. --- allows for conditional heteroskedasticity!
\end{frame}

% ----------------------------------------------------------------------------
\begin{frame}{MDS Properties}
    \begin{theorem}
        Let $\{Y_t\}$ be MDS w.r.t. $\{\F_t\}$, and let $g_{t-1} = g(Y_{t-1}, Y_{t-2}, \ldots)$ be measurable and integrable. Then:
        \begin{enumerate}
            \item $\{Y_t g_{t-1}\}$ is also MDS
            \item $Y_t$ and $g_{t-1}$ are uncorrelated
        \end{enumerate}
    \end{theorem}
    
    \vspace{0.5em}
    
    \textbf{Proof sketch}:
    \[
    \E[Y_t g_{t-1} | \F_{t-1}] = g_{t-1} \E[Y_t | \F_{t-1}] = g_{t-1} \cdot 0 = 0
    \]
    
    \vspace{1em}
    
    \textbf{Implication}: Products of MDS with past functions remain MDS.
    
    This is crucial for:
    \begin{itemize}
        \item Deriving variance formulas
        \item Establishing CLTs for time series
        \item Analyzing ARCH/GARCH models
    \end{itemize}
\end{frame}

% ----------------------------------------------------------------------------
\begin{frame}{Proof: Autocovariance Symmetry}
    \begin{lemma}
        For a weakly stationary process: $\Gamma_k = \Gamma_{-k}'$
    \end{lemma}
    
    \begin{proof}
        Since the process is weakly stationary:
        \[
        \E[Y_t] = \E[Y_{t-k}] = \mu \quad \forall k
        \]
        
        Therefore:
        \begin{align*}
            \Gamma_k &= \E[(Y_t - \mu)(Y_{t-k} - \mu)'] \\
            &= \E[(Y_{t-k} - \mu)(Y_t - \mu)']' \\
            &= \Gamma_{-k}'
        \end{align*}
    \end{proof}
\end{frame}

% ----------------------------------------------------------------------------
\begin{frame}{Detailed Derivation: Monetary Transmission VAR}
    Starting from:
    \[
    B\mathbf{w}_t = \mathbf{a}_0 + A_1 \mathbf{w}_{t-1} + \ubf_t
    \]
    
    With $B = \begin{bmatrix} 1 & -\lambda \\ 0 & 1 \end{bmatrix}$, we have $B^{-1} = \begin{bmatrix} 1 & \lambda \\ 0 & 1 \end{bmatrix}$
    
    \vspace{0.5em}
    
    Multiplying through:
    \[
    \mathbf{w}_t = B^{-1}\mathbf{a}_0 + B^{-1}A_1 \mathbf{w}_{t-1} + B^{-1}\ubf_t
    \]
    
    \vspace{0.5em}
    
    \textbf{Reduced-form parameters}:
    \begin{align*}
        \mathbf{c}_0 &= \begin{bmatrix} \lambda\gamma(i^* - \rho\pi^*) \\ \gamma(i^* - \rho\pi^*) \end{bmatrix} \\
        C_1 &= \begin{bmatrix} 1 + \lambda\gamma(\rho-1) & 0 \\ \gamma(\rho-1) & 0 \end{bmatrix} \\
        \mathbf{v}_t &= \begin{bmatrix} u_{1t} + \lambda u_{2t} \\ u_{2t} \end{bmatrix}
    \end{align*}
\end{frame}

% ----------------------------------------------------------------------------
\begin{frame}{Additional Mixing Conditions}
    Beyond $\alpha$-mixing, other conditions exist:
    
    \vspace{0.5em}
    
    \textbf{$\beta$-mixing} (absolute regularity):
    \[
    \beta_m = \sup_t \E\left[\sup_{H \in \F_{t+m}^{\infty}} |\P(H | \F_{-\infty}^t) - \P(H)|\right]
    \]
    
    \textbf{$\rho$-mixing} (maximal correlation):
    \[
    \rho_m = \sup_t \sup_{\substack{f \in L^2(\F_{-\infty}^t) \\ g \in L^2(\F_{t+m}^{\infty})}} |\text{Corr}(f, g)|
    \]
    
    \vspace{0.5em}
    
    \textbf{Hierarchy}:
    \[
    \text{i.i.d.} \Rightarrow \beta\text{-mixing} \Rightarrow \alpha\text{-mixing}
    \]
    \[
    \beta\text{-mixing} \Rightarrow \rho\text{-mixing}
    \]
    
    Each has different implications for CLTs and moment inequalities.
\end{frame}

% ----------------------------------------------------------------------------
\begin{frame}{References}
    \textbf{Textbooks}:
    \begin{itemize}
        \item Hamilton (1994): \textit{Time Series Analysis}
        \item Brockwell \& Davis (1991): \textit{Time Series: Theory and Methods}
        \item Lütkepohl (2005): \textit{New Introduction to Multiple Time Series Analysis}
    \end{itemize}
    
    \vspace{0.5em}
    
    \textbf{Course Materials}:
    \begin{itemize}
        \item Lecture Notes: Sections 7.1--7.3
        \item Cochrane (2005): \textit{Time Series for Macroeconomics and Finance}
    \end{itemize}
    
    \vspace{0.5em}
    
    \textbf{Classic Papers}:
    \begin{itemize}
        \item Samuelson (1939): ``Interactions between the multiplier analysis and the principle of acceleration''
        \item Sims (1980): ``Macroeconomics and Reality''
    \end{itemize}
\end{frame}

\end{document}

