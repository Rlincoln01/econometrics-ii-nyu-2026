\documentclass[12pt]{article}
 
\usepackage[margin=1in]{geometry} 
\usepackage{amsmath,amsthm,amssymb,scrextend}
\usepackage{fancyhdr}
\usepackage{graphicx}
\usepackage{enumitem}
\pagestyle{fancy}
\setlength{\headheight}{28pt}

% Red "alertblock"-style boxes (Beamer-like)
\usepackage{xcolor}
\usepackage[most]{tcolorbox}
\newtcolorbox{alertblock}[1]{%
  colback=red!5!white,
  colframe=red!70!black,
  fonttitle=\bfseries,
  title=#1,
  sharp corners,
  boxrule=0.9pt,
  left=1.2mm,right=1.2mm,top=1mm,bottom=1mm
}

% Common shortcuts (kept close to template)
\newcommand{\R}{\mathbb R}
\newcommand{\N}{\mathbb N}
\newcommand{\Z}{\mathbb Z}
\newcommand{\E}{\mathbb E}
\newcommand{\Var}{\mathrm{Var}}
\newcommand{\imp}{\Rightarrow}
\newcommand{\set}[1]{\left\{ #1 \right\}}

\usepackage{datetime}

% Date (update as needed)
\newdate{date}{06}{02}{2026}

\usepackage[colorlinks=true, pdfstartview=FitV, linkcolor=blue,
citecolor=blue, urlcolor=blue]{hyperref}

\newtheorem*{definition}{Definition}
\newtheorem{problem}{Problem}
\newtheorem{exercise}{Exercise}

\begin{document}
 
\lhead{TA: Rafael Lincoln \\ Prof. Tim Cogley}
\rhead{Econometrics II (Time Series) \\ For: \displaydate{date}}

\section*{Problem Set 2 --- State space, trends, and policy regime change}

\begin{alertblock}{Important}
We \textbf{do not collect} this problem set, and you \textbf{do not need to do all exercises}.
We will cover the solutions in class and provide written solutions afterwards for your study.
The goal is to give you more material to prepare for the exam.
\end{alertblock}

\vspace{0.8em}

\begin{problem}[Long-run risk: state-space and likelihood]
Consider a version of Bansal and Yaron's (2004) \textit{long-run risk} model,
\begin{align*}
x_t &= \rho x_{t-1} + \sigma_\varepsilon \varepsilon_t,\\
g_t &= x_t + \sigma_\eta \eta_t,
\end{align*}
where $(\varepsilon_t,\eta_t)'$ is an i.i.d.\ standard normal random vector. The free parameters are
\[
\theta = [\rho,\sigma_\varepsilon,\sigma_\eta]'.
\]
\begin{enumerate}
    \item[(a)] Derive a VAR representation for $(x_t,g_t)'$. What condition on $\theta$ is necessary for covariance stationarity?
    \item[(b)] Assuming that the stationarity condition is satisfied and that $x_t$ and $g_t$ are both observable, explain how you would estimate $\theta$ by maximum likelihood.
    \item[(c)] Now assume that $g_t$ is observable but $x_t$ is not. Assuming stationarity, explain how you would estimate $\theta$ by maximum likelihood.
\end{enumerate}
\end{problem}

\vspace{0.6em}

\begin{problem}[Beveridge--Nelson stochastic trend via a VAR(2)]
Beveridge and Nelson (1981) define a stochastic trend $\tau_t$ as the level to which a series $x_t$ is expected to converge in the long run,
\[
\tau_t \;=\; x_t + E_t \sum_{j=1}^{\infty}(\Delta x_{t+j}-\mu_x),
\]
where $\mu_x$ is the unconditional mean of $\Delta x_t$. Suppose that $y_t=[\Delta x_t,w_t']'$, where $w_t$ is a vector of variables that might be helpful for predicting $\Delta x_t$. Assume that $y_t$ is covariance stationary and for simplicity that its unconditional mean is zero. Furthermore, suppose that $y_t$ is well approximated by a VAR(2) representation with homoskedastic gaussian innovations.
\begin{enumerate}
    \item[(a)] Explain how you choose variables to include in $w_t$.
    \item[(b)] How would you estimate the VAR parameters by MLE?
    \item[(c)] Derive an expression for the trend $\tau_t$ in terms of the VAR parameters and current and lagged values of $y_t$.
    \item[(d)] How would you approximate the asymptotic distribution of the coefficients mapping current and lagged values of $y_t$ into $\tau_t$?
    \item[(e)] Prove that $\tau_t$ is a random walk (with drift if $\mu_{\Delta x}\neq 0$).
\end{enumerate}
\end{problem}

\vspace{0.6em}

\begin{problem}[Clarida--Gal\'{\i}--Gertler (1999)]\label{ex:NK_to_AR1_policybreak}
\begin{figure}[h!]
    \centering
    \includegraphics[width=0.85\textwidth]{figures/ffs_inf.png}
    \caption{Federal funds rate and inflation (see Clarida, Gal\'{\i}, and Gertler, 1999).}
\end{figure}

\noindent A common interpretation of U.S.\ monetary history is that policy became more aggressive against inflation from the Burns/Miller era to the Volcker era. In a structural New Keynesian (NK) model, changes in the Taylor-rule parameter can alter the reduced-form persistence of inflation, which is closely related to the Lucas critique. Consider the 3-equation NK model:
\begin{align}
\text{(IS)}\qquad 
x_t &= \E_t x_{t+1}-\frac{1}{\sigma}\Big(i_t-\E_t \pi_{t+1}-r_t^n\Big), \label{eq:IS}\\
\text{(NKPC)}\qquad 
\pi_t &= \beta\,\E_t \pi_{t+1}+\kappa x_t+u_t, \label{eq:NKPC}\\
\text{(Taylor)}\qquad 
i_t &= r_t^n+\phi_\pi \pi_t, \qquad \phi_\pi>0, \label{eq:TR}
\end{align}
where $x_t$ is the output gap, $\pi_t$ inflation, $i_t$ the nominal interest rate, and $u_t$ a cost-push shock. The discount factor satisfies
\[
\beta=\frac{1}{1+\rho},\qquad \rho>0.
\]
Assume throughout:
\begin{enumerate}[label=(A\arabic*)]
\item $r_t^n=r^n$ is constant;
\item $\E_t x_{t+1}=0$;
\item $\E_t \pi_{t+1}=\pi_{t-1}$;
\item $u_t$ is i.i.d.\ with $\E[u_t]=0$ and $\Var(u_t)=\sigma_u^2$.
\end{enumerate}

\begin{enumerate}[label=(\alph*)]
\item \textbf{Derive the reduced-form AR(1) for inflation and relate it to policy.}
Eliminate $x_t$ and $i_t$ in \eqref{eq:IS}--\eqref{eq:TR} and show that inflation can be written as an AR(1):
\[
\pi_t = a(\phi_\pi,\kappa,\rho)\,\pi_{t-1}+\varepsilon_t,
\]
for some innovation $\varepsilon_t$. Give explicit expressions for $a(\phi_\pi,\kappa,\rho)$ and $\varepsilon_t$.
Then determine the sign of $\partial a/\partial \phi_\pi$ and interpret it (one sentence).

\item \textbf{Burns vs.\ Volcker: structural hypothesis stated as a reduced-form restriction.}
Let $T_b$ be a break date (e.g.\ 1979:Q3). Assume $(\beta,\kappa,\sigma,\sigma_u^2,r^n)$ are constant over time, but policy may change:
\[
\phi_\pi=
\begin{cases}
\phi_{\pi,1}, & t\le T_b,\\
\phi_{\pi,2}, & t> T_b.
\end{cases}
\]
Using your mapping from part (a), translate ``policy changed from Burns to Volcker'' into a null and alternative hypothesis in terms of the reduced-form persistence parameters $a_1$ and $a_2$.

\item \textbf{Reduced-form likelihood test with dependent data.}
Assume the reduced form holds with a break in the AR(1) coefficient:
\[
\pi_t = c + a_1\pi_{t-1}+\varepsilon_t,\quad t\le T_b,
\qquad
\pi_t = c + a_2\pi_{t-1}+\varepsilon_t,\quad t> T_b,
\]
with
\[
\varepsilon_t\mid\mathcal{F}_{t-1}\sim\mathcal{N}(0,\sigma^2),\qquad 
\mathcal{F}_{t-1}=\sigma(\pi_{t-1},\pi_{t-2},\ldots).
\]
\begin{enumerate}[label=(\roman*)]
\item Write the conditional log-likelihood $\ell_n(\theta)$ conditioning on $\pi_0$,\\
where $\theta=(c,a_1,a_2,\sigma^2)$.
\item Describe an LR test of $H_0:a_1=a_2$ and state its asymptotic null distribution.
\end{enumerate}

\item \textbf{Wald and LM versions of the same restriction.}
Let $r(\theta)=a_1-a_2$.
\begin{enumerate}[label=(\roman*)]
\item Write the Wald statistic for testing $r(\theta)=0$ using the unrestricted MLE.
\item Write the LM statistic in the standard score--information form evaluated at the restricted MLE (no need to simplify fully).
\end{enumerate}

\item \textbf{Interpretation.}
If you estimate $\hat a_2 < \hat a_1$ and reject $H_0:a_1=a_2$, interpret the result in terms of $\phi_{\pi,2}$ vs.\ $\phi_{\pi,1}$ and monetary policy aggressiveness.

\item \textbf{Lucas critique (short).}
Explain why an AR(1) estimated under the Burns regime may fail to forecast inflation under the Volcker regime. In your answer, identify which objects are ``structural'' and which are ``reduced-form'' in this problem.
\end{enumerate}
\end{problem}

\vspace{0.6em}

\begin{problem}[Kalman filter with correlated state and measurement innovations]
Consider a state-space model with correlated state and measurement innovations,
\[
S_t = AS_{t-1}+B\varepsilon_{1t},
\qquad
X_t = CS_t + D\varepsilon_{2t},
\]
where $X_t$ is an $m\times 1$ vector of observables and $S_t$ is a $n\times 1$ vector that may include latent variables. The innovations $\varepsilon_{1t}$ and $\varepsilon_{2t}$ have dimensions $p\times 1$ and $q\times 1$, respectively, and are i.i.d.\ normal with mean zero and covariance matrix
\[
E\left(
\begin{bmatrix}
\varepsilon_{1t}\\
\varepsilon_{2t}
\end{bmatrix}
[\varepsilon'_{1t},\varepsilon'_{2t}]
\right)
=
\begin{bmatrix}
Q & F\\
F' & R
\end{bmatrix}.
\]
Derive the Kalman filter for this model.
\end{problem}

\vspace{0.6em}

\begin{problem}[ARMA(2,2) $\rightarrow$ state space]
Suppose $y_t$ has an ARMA(2,2) representation
\[
(1-\rho_1 L-\rho_2 L^2)y_t = (1+\theta_1 L+\theta_2 L^2)\varepsilon_t,
\]
where $\varepsilon_t$ is i.i.d.\ $N(0,\sigma^2)$. Put this into state-space form and verify your solution.
\end{problem}

\end{document}

